\chapter{Basics of Circuits}

\section{The direction of current and voltage}

In complex problems, we can not always know the direction of currents or voltage. The usual solution is to assume a direction, and use it to solve problems. If the solution of current or voltage, is positive, then the position is just what we assumed, and vice versa.

\section{How to determine whether a component is consuming or providing energy}

For resistors, resistors are always consuming energy.

For power sources, if the direction of current is from the positive electrode to the negative electrode, then the power source is consuming energy, and vice versa.

\section{Eletcronic Components}

\subsection{Resistors}

\begin{equation*}
  \begin{aligned}
    U = - I R
  \end{aligned}
\end{equation*}

\subsection{Power Sources}

\subsubsection{(Controlled) Voltage Source}

\begin{equation*}
  \begin{aligned}
    P = U I
  \end{aligned}
\end{equation*}

The resistance of an ideal voltage source is : $0$

Note that an ideal voltage source must not be short-circuited.

\subsubsection{(Controlled) Current Source}

The current through a current source is only decided by the source itself.

The resistance of an ideal current source is : $\infty$

Note that an ideal current source must not be open-circuited.

\section{Kirchhoff's Laws}

\begin{itemize}
\item branch
\item node
\item loop
\item mesh
  
\end{itemize}

\subsection{Kirchhoff's Current Law}

For each node in the circuit, as the node can not accumulate charges, the sum of current is zero.

\begin{equation*}
  \begin{aligned}
    \sum I = 0
  \end{aligned}
\end{equation*}

\subsection{Kirchhoff's Voltage Law}

For each loop in circuit, the sum of the voltage in of all branches is zero.

\begin{equation*}
  \begin{aligned}
    \sum U = 0
  \end{aligned}
\end{equation*}

\section{Gain of an Amplifier Circuit}

\begin{table*}[htbp]
  \centering
  \begin{tabular}{|Sc|Sc|Sc|Sc|Sc|Sc|}
    \hline
    & Voltage & Current & Transresistance & Transconductance & Power \\
    \hline
    Gain & $A_v = \dfrac{v_o}{v_i}$ & $A_i = \dfrac{i_o}{i_i}$ & $A_r = \dfrac{v_o}{i_i}$ & $A_g = \dfrac{i_o}{v_i}$ & $A_p = \dfrac{P_o}{P_i}$ \\
    \hline
    Gain in dB & $20\lg \left| A_v \right|$ & $20\lg \left| A_i \right|$ & $20\lg \left| A_r \right|$ & $20\lg \left| A_g \right|$ & $10\lg A_p$ \\
    \hline
  \end{tabular}
\end{table*}


%%% Local Variables:
%%% mode: latex
%%% TeX-master: "Analogue_Electronics"
%%% End:
