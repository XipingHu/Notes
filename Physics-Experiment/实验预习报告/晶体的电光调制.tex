\documentclass{article}

% Chinese Support using xeCJK
% \usepackage{xeCJK}
% \setCJKmainfont{SimSun}

% Chinese Support using CTeX
\usepackage{ctex}

% Math Support
\usepackage{amsmath}
\usepackage{amsfonts}
\usepackage{amssymb}
\usepackage{wasysym}
\newcommand{\angstrom}{\text{\normalfont\AA}}
\usepackage{fancyhdr}

% Graphics Support
\usepackage{graphicx}
\usepackage{float}

% Reduced page margin
\usepackage{geometry}
\geometry{a4paper,scale=0.8}

\usepackage{caption}
\usepackage{subcaption}

% d and e should be math operators
\newcommand*{\dif}{\mathop{}\!\mathrm{d}}
\newcommand*{\md}{\mathop{}\!\mathrm{d}}
\newcommand*{\me}{\mathrm{e}}

% No indent for each paragraph
% \usepackage{parskip}
% \setlength{\parindent}{0cm}

% Bold style for Greek letters
\usepackage{bm}
\let\Oldmathbf\mathbf
\renewcommand{\mathbf}[1]{\boldsymbol{\Oldmathbf{#1}}}

% More space for dfrac in cell
\usepackage{cellspace}
\setlength{\cellspacetoplimit}{5pt}
\setlength{\cellspacebottomlimit}{5pt}

% SI units
\newcommand{\si}[1]{\  \mathrm{#1}}

% Multi-line author information
\usepackage{authblk}
\author{物理(4+4)1801 \quad  胡喜平 \quad U201811966}
\affil{个人网站 https://hxp.plus/ \quad 电子邮件 hxp201406@gmail.com}

\title{综合物理实验预习笔记——光电传感器综合实验}

\pagestyle{fancy}
\fancyhf{}
\lhead{源码地址:https://github.com/hxp-plus/Notes/tree/master/Physics-Experiment}
\rfoot{第 \thepage 页}
\renewcommand{\headrulewidth}{1pt}
\renewcommand{\footrulewidth}{1pt}

\begin{document}

\maketitle\thispagestyle{fancy}

\section{实验内容}

\begin{itemize}
\item 测量\textbf{光敏电阻}的伏安特性曲线和光照特性曲线。
\item 测量\textbf{光敏二极管}的伏安特性曲线和光照特性曲线。
\item 测量\textbf{硅光电池}的伏安特性曲线和光照特性曲线。
\item 测量\textbf{光敏三极管}的伏安特性曲线和光照特性曲线。
\end{itemize}

\section{实验原理}

\subsection{光敏电阻}

当光照射到光敏电阻上时,价电子迁移到导带,价带中留下空穴,导致电导率发生改变。电导率的变化为

\begin{equation*}
  \begin{aligned}
    \Delta \sigma = \Delta p \cdot e \cdot \mu_p + \Delta n \cdot e \cdot \mu_n
  \end{aligned}
\end{equation*}

其中$\Delta p$是空穴浓度,$\Delta n$是电子浓度,$e$是电子电量,$\Delta \sigma$是电导率的变化,其余为常数。

因此在没有光照的情况下,光敏电阻的电阻很大,有光照的情况下,光敏电阻的电阻小。在有光照的情况下,加电压生成\textbf{光电流}。

\begin{equation*}
  \begin{aligned}
    I_{ph} = \dfrac{A}{d} \cdot \Delta \sigma \cdot U 
  \end{aligned}
\end{equation*}

光照强度一定时,光电流和电压呈正比。电压一定时,光照强度越大,光电流越大。但是光照强度和光电流不是线性关系,逐渐增大光照强度时,初期光电流迅速增加,后期光电流增加缓慢。

\subsection{硅光电池}

硅光电池工作时,需要零偏或者反偏。当加入反偏电压$V$时

\begin{equation*}
  \begin{aligned}
    I = I_s \left[ \exp \left( \dfrac{eV}{kT}  \right) - 1 \right] + I_p
  \end{aligned}
\end{equation*}

$I_s$是饱和电流,$I_p$是光电流。当$V=0$时,$I=I_p$。实验中$V>0$,$I=I_s - I_p$。其中光电流与光的功率的关系为

\begin{equation*}
  \begin{aligned}
    I_p = R P_i
  \end{aligned}
\end{equation*}

$P_i$为光的功率。

硅光电池的\textbf{短路电压}、\textbf{短路电流}为光电池直接串联电压表或电流表时测得的电压和电流。硅光电池的\textbf{负载特性}为:低负载时电流大电压小,高负载时电流小电压大。

\subsection{光敏二极管与三极管}

在没有光照的条件下,光敏二极管和三极管的\textbf{饱和反向漏电流}小,称为暗电流。在有光的条件下,\textbf{饱和反向漏电流}大,且会随着电阻变化。此时光电流与偏压的关系成为伏安特性。

\end{document} 
