\documentclass{article}
% Chinese
% \documentclass[UTF8, nofonts, mathptmx, 12pt, onecolumn]{article}
% \usepackage{xeCJK}
% \setCJKmainfont{SimSun}
\usepackage{amsmath}
\usepackage{amsfonts}
\usepackage{amssymb}
\usepackage{wasysym}
% \usepackage{ctex}
\usepackage{graphicx}
\usepackage{float}
\usepackage{geometry}
\geometry{a4paper,scale=0.8}
\usepackage{caption}
\usepackage{subcaption}
% \newcommand{\oiint}{\mathop{{\int\!\!\!\!\!\int}\mkern-21mu \bigcirc} {}}
\newcommand*{\dif}{\mathop{}\!\mathrm{d}}
\newcommand*{\md}{\mathop{}\!\mathrm{d}}
\newcommand*{\me}{\mathrm{e}}

\usepackage{parskip}
\setlength{\parindent}{0cm}

\usepackage{bm}
\let\Oldmathbf\mathbf
\renewcommand{\mathbf}[1]{\boldsymbol{\Oldmathbf{#1}}}
\let\eqnarray\align

\author{Xiping Hu}
\usepackage{authblk}
\author{Xiping Hu}
\affil{http://thehxp.tech/}
\title{Homework for Chapter II}

\begin{document}
\maketitle

\paragraph{Q1}

Assume $\vec{A} = x \left( z-y \right) \hat{x} + y \left( x - z \right) \hat{y} + z \left( y - x \right) \hat{z}$, Solve the rotation of $\vec{A}$ at $M \left( 1,0,1 \right)$ and the circulation density along $\vec{n} = 2 \hat{x} + 6 \hat{y} + 3 \hat{z}$

\paragraph{Solution}

The rotation of $\vec{A}$ is
\begin{equation*}
  \begin{aligned}
    \mathrm{rot} \vec{A} = \nabla \times \vec{A} = \left|
      \begin{array}{ccc}
        \hat{x}& \hat{y} &\hat{z} \\
       \dfrac{\partial}{\partial x}  &  \dfrac{\partial}{\partial y} & \dfrac{\partial}{\partial z} \\
       x \left( z - y \right) & y \left( x - z \right) & z \left( y - x \right)
      \end{array}
    \right| =
    \left( y + z \right) \hat{x} + \left( x + z \right) \hat{y} + \left( x + y \right) \hat{z}
  \end{aligned}
\end{equation*}
\begin{equation*}
  \begin{aligned}
    \left. \mathrm{rot} \vec{A} \right| _{\left( 1,0,1 \right)} = \hat{x} + 2 \hat{y} + \hat{z}
  \end{aligned}
\end{equation*}

The circulation density along $\vec{n} = 2 \hat{x} + 6 \hat{y} + 3 \hat{z}$ is
\begin{equation*}
  \begin{aligned}
    \mathrm{rot} \vec{A} \cdot \hat{n} = \left( 1,2,1 \right) \cdot \dfrac{1}{7} \left( 2,6,3 \right) = \dfrac{17}{7} 
  \end{aligned}
\end{equation*}

\paragraph{Q2}

The speed of light is $3 \times 10^8 \ \mathrm{m/s} $. What is the wavelength of a red light, whose frequency is $5 \times 10^{4} \ \mathrm{Hz}$? Compare your result with a $60 \ \mathrm{Hz}$ EM wave.

\paragraph{Solution}

The wavelength of the red light is

\begin{equation*}
  \begin{aligned}
    \lambda = \dfrac{3 \times 10^8}{5 \times 10^{4}} = 600 \ \mathrm{nm}
  \end{aligned}
\end{equation*}

The wavelength of the $60 \ \mathrm{Hz}$ EM wave is

\begin{equation*}
  \begin{aligned}
    \lambda = \dfrac{3 \times 10^8}{60} = 5 \times 10^{6} \ \mathrm{m}
  \end{aligned}
\end{equation*}

\paragraph{Q3}

Two wave functions:
$\psi_1 = 4 \cos \left[ 2 \pi \left( 0.2 x - 3 t \right) \right]$,
$\psi_2 = \cos \left( 7 x + 3.5 t \right) / 2.5$
Calculate the frequency, wavelength, period, amplitude and phase velocity for each function.

\paragraph{Solution}
For 
$\psi_1 = 4 \cos \left[ 2 \pi \left( 0.2 x - 3 t \right) \right] = 4 \cos \left[ 10 \pi \left( x - 15 t \right) \right]$
\begin{equation*}
  \begin{aligned}
    f &= \dfrac{v}{\lambda} = 75 \ \mathrm{Hz}\\
    v &= 15 \ \mathrm{m/s} \\ 
    \lambda &= \dfrac{2\pi}{k} = 0.2 \ \mathrm{m} \\
    A &= 4 \ \mathrm{m} \\
    T &= \dfrac{\lambda}{v} = 0.0133 \ \mathrm{s}
  \end{aligned}
\end{equation*}

For 
$\psi_2 = \cos \left( 7 x + 3.5 t \right) / 2.5 = \cos \left[ 2.8 \left( x + 1.4 t \right) \right]$
\begin{equation*}
  \begin{aligned}
    f &= \dfrac{v}{\lambda} = 0.625 \ \mathrm{Hz}\\
    v &= \dfrac{3}{0.2} = 1.4 \ \mathrm{m/s} \\ 
    \lambda &= \dfrac{2\pi}{k} = 2.24 \ \mathrm{m} \\
    A &= 1 \ \mathrm{m} \\
    T &= \dfrac{\lambda}{v} = 1.60 \ \mathrm{s}
  \end{aligned}
\end{equation*}

\paragraph{Q4}

Verify that the following functions are solutions of wave function:
\begin{equation*}
  \begin{aligned}
    \psi_1 \left( x,t \right) &= A \exp \left[ i \left( k x - \omega t \right) \right]\\
    \psi_2 \left( x,t \right) &= A \exp \left[ i \left( - k x - \omega t \right)\right] \\
    \psi_3 \left( x,t \right) &= A \exp \left[ i \left( k x - \omega t \right)\right] + B \exp \left[ i \left( - k x - \omega t \right) \right]
  \end{aligned}
\end{equation*}

\paragraph{Solution}

The form of one dimensional wave function is

\begin{equation*}
  \begin{aligned}
    \dfrac{\partial^2 \psi}{\partial x^{2}} = - \dfrac{1}{v^{2}} \dfrac{\partial^{2} \psi}{\partial t^{2}} 
  \end{aligned}
\end{equation*}

We now insert $\psi_1$ into the wave function

\begin{equation*}
  \begin{aligned}
     - A k^2 \exp \left[ i \left( k x - \omega t \right) \right] = - A \dfrac{\omega^2 }{v^{2}} \exp \left[ i \left( k x - \omega t \right)\right]
  \end{aligned}
\end{equation*}

So that $\psi_1$ is a wave function if $v = \omega / k$

Now we insert $\psi_2$ into the wave function, similarly,


\begin{equation*}
  \begin{aligned}
     - A k^2 \exp \left[ i \left( k x - \omega t \right) \right] = - A \dfrac{\omega^2 }{v^{2}} \exp \left[ i \left( k x - \omega t \right)\right]
  \end{aligned}
\end{equation*}

So that $\psi_2$ is a wave function if $v = \omega / k$

Now we insert $\psi_3$ into the wave function, similarly,

\begin{equation*}
  \begin{aligned}
     - A k^2 \exp \left[ i \left( k x - \omega t \right) \right] - A k^2 \exp \left[ i \left( k x - \omega t \right) \right] = - A \dfrac{\omega^2 }{v^{2}} \exp \left[ i \left( k x - \omega t \right)\right]- A \dfrac{\omega^2 }{v^{2}} \exp \left[ i \left( k x - \omega t \right)\right]
  \end{aligned}
\end{equation*}

$\psi_3$ is a wave function if $v = \omega / k$

\paragraph{Q4}

Write the Maxwell Equation in 3 dimensions. Derive the wave function of in all components.

\paragraph{Solution}

The differentiation form of Maxwell's equation is

\begin{equation*}
  \begin{aligned}
    \nabla \times \vec{E} &= - \dfrac{\partial \vec{B}}{\partial t} \\
    \nabla \times \vec{B} &= \mu_0 \epsilon_0 \dfrac{\partial \vec{E}}{\partial t} \\
    \nabla \cdot \vec{B} &= 0 \\
    \nabla \cdot \vec{E} &= 0
  \end{aligned}
\end{equation*}

We expand the equation into 3 components along the 3 axis

\begin{equation*}
  \begin{aligned}
    \dfrac{\partial E_{z}}{\partial y} - \dfrac{\partial E_{y}}{\partial z} =& - \dfrac{\partial B_{x}}{\partial t} \\
    \dfrac{\partial E_{x}}{\partial z} - \dfrac{\partial E_{z}}{\partial x} =& - \dfrac{\partial B_{y}}{\partial t} \\
    \dfrac{\partial E_{y}}{\partial x} - \dfrac{\partial E_{x}}{\partial y} =& - \dfrac{\partial B_{z}}{\partial t} \\
    \dfrac{\partial B_{z}}{\partial y} - \dfrac{\partial B_{y}}{\partial z} =& \mu_0 \epsilon_0 \dfrac{\partial E_{x}}{\partial t} \\
    \dfrac{\partial B_{x}}{\partial z} - \dfrac{\partial B_{z}}{\partial x} =& \mu_0 \epsilon_0 \dfrac{\partial E_{y}}{\partial t} \\
    \dfrac{\partial B_{y}}{\partial x} - \dfrac{\partial B_{x}}{\partial y} =& \mu_0 \epsilon_0 \dfrac{\partial E_{z}}{\partial t} \\
    \dfrac{\partial E_{x}}{\partial x} + \dfrac{\partial E_{y}}{\partial y} +& \dfrac{\partial E_{z}}{\partial z} = 0 \\
    \dfrac{\partial B_{x}}{\partial x} + \dfrac{\partial B_{y}}{\partial y} +& \dfrac{\partial B_{z}}{\partial z} = 0 \\
  \end{aligned}
\end{equation*}

Assume $\vec{E}$ is along the $\hat{x}$ direction. Therefore $\vec{B}$ is along the $\hat{y}$ direction. If $\vec{E}$ and $\vec{B}$ is not in these 2 directions, we can rotate the axis.

So we have

\begin{equation*}
  \begin{aligned}
    E_y =& 0 \\
    E_z =& 0 \\
    B_x =& 0 \\
    B_z =& 0 
  \end{aligned}
\end{equation*}

Then

\begin{equation*}
  \begin{aligned}
    \dfrac{\partial E_{x}}{\partial z} =& - \dfrac{\partial B_{y}}{\partial t} \\
    - \dfrac{\partial E_{x}}{\partial y} =& 0 \\
    - \dfrac{\partial B_{y}}{\partial z} =& \mu_0 \epsilon_0 \dfrac{\partial E_{x}}{\partial t} \\
    \dfrac{\partial B_{y}}{\partial x} =& 0 \\
    \dfrac{\partial E_{x}}{\partial x} =& 0 \\
    \dfrac{\partial B_{y}}{\partial y} =& 0 \\
  \end{aligned}
\end{equation*}

Finally

\begin{equation*}
  \begin{aligned}
    \dfrac{\partial^{2} B_{y}}{\partial z^{2}} =& - \mu_0 \epsilon_0 \dfrac{\partial^{2} E_{x}}{\partial t \partial z} = \mu_0 \epsilon_0 \dfrac{\partial^{2} B_{y}}{\partial t^{2}}\\
    \dfrac{\partial^2 E_x}{\partial z^2} &= - \dfrac{\partial^2 B_y}{\partial t \partial z} =   \mu_0 \epsilon_0 \dfrac{\partial^2 E_x}{\partial t^2} 
  \end{aligned}
\end{equation*}

\paragraph{Q6}

A plane wave on y-z plane, traveling along $\vec{r}$. The angle between $\vec{r}$ and y-axis is $\theta$. The initial phase is $0$. Find the complex amplitude of this wave.

\paragraph{Solution}

\begin{equation*}
  \begin{aligned}
    \vec{E} \left( \vec{r} , t \right) &= \vec{A} \exp \left( i \vec{k} \cdot \vec{r} \right) \exp \left( - i \omega t \right) \\
    \vec{r} &= x \hat{x} + y \hat{y} + z \hat{z} \\
    \hat{k} &= k \left( \hat{y} \cos \theta + \hat{z} \sin \theta \right)
  \end{aligned}
\end{equation*}

\paragraph{Q7}

A electromagnetic wave, $E_x = 0$, $E_y = 2 \cos \left[ 2 \pi \times 10^{14} \left( z/c -t \right) \right]$, $E_z = 0$. What is the amplitude, wavelength, frequency of the wave? How is $\vec{B}$ exists?

\paragraph{Solution}

\begin{equation*}
  \begin{aligned}
    E_y = 2 \cos \left[ 2 \pi \times 10^{14} c \left( z - ct \right) \right]
  \end{aligned}
\end{equation*}

\begin{equation*}
  \begin{aligned}
    \lambda &= 3.3 \times 10^{-21} \\
    A &= 2 \\
    f &= c/\lambda = 9.09 \times 10^{28} \\
    B &= c E
  \end{aligned}
\end{equation*}




\end{document}