\chapter{Electromagnetic Theory and photons}

\section{Longitudinal and Transverse}

\begin{itemize}
\item Longitudinal: medium is in the direction of motion of wave.
\item Transverse: medium is perpendicular to the motion of wave.
  
\end{itemize}

\section{Wave Equation}

\begin{equation*}
  \begin{aligned}
    \psi \left( x, t \right) = f \left( x + v t \right) \\
  \end{aligned}
\end{equation*}

\begin{equation*}
  \left\{
  \begin{aligned}
    \dfrac{\partial}{\partial x} &= \dfrac{\partial}{\partial \left( x + v t \right)} \cdot \dfrac{\partial \left( x + v t \right)}{\partial x} = \dfrac{\partial}{\partial \left( x + v t \right)} \\   
    \dfrac{\partial}{\partial t} &= \dfrac{\partial}{\partial \left( x + v t \right)} \cdot \dfrac{\partial \left( x + v t \right)}{t} = v \cdot \dfrac{\partial}{\partial \left( x + v t \right)}     
  \end{aligned}
  \right.
  \quad \Rightarrow \quad  
  \left\{
  \begin{aligned}
    \dfrac{\partial^2}{\partial x^2} &= \dfrac{\partial^2}{\partial \left( x + v t \right)^2} \\
    \dfrac{\partial^2}{\partial t^2} &= v^2 \cdot \dfrac{\partial^2}{\partial \left( x + v t \right)^2}  
  \end{aligned}
  \right.
  \quad \Rightarrow \quad  
  \left\{
  \begin{aligned}
    \dfrac{\partial^2 \psi}{\partial x^2} &= \dfrac{\partial^2 \psi}{\partial \left( x + v t \right)^2} \\
    \dfrac{\partial^2 \psi}{\partial t^2} &= v^2 \cdot \dfrac{\partial^2 \psi}{\partial \left( x + v t \right)^2}  
  \end{aligned}
  \right.
\end{equation*}

\begin{equation*}
  \begin{aligned}
    \Rightarrow \dfrac{\partial^2 \psi}{\partial x^2} = \dfrac{1}{v^2} \cdot \dfrac{\partial^2 \psi}{\partial t^2}   
    \quad \Rightarrow \quad
    \nabla^2 \psi \left( x,y,z \right) = \dfrac{1}{v^2} \cdot \dfrac{\partial^2 \psi \left( x,y,z \right)}{\partial t^2}  
  \end{aligned}
\end{equation*}

\section{Maxwell's Equation}

Faraday's Induction Law

\begin{equation*}
  \oint_{c} \vec{E} \cdot \md \vec{l} = - \dfrac{\md}{\md t} \iint \vec{B} \cdot \md \vec{S}
  \quad \Rightarrow \quad
  \nabla \times \vec{E} = - \dfrac{\partial \vec{B}}{\partial t}    
\end{equation*}

Gauss's Law

\begin{equation*}
  \oiint_{A} \vec{E} \cdot \md \vec{S} = \dfrac{1}{\varepsilon_{0}} \iiint_{v} \rho \md V 
  \quad \Rightarrow \quad
  \nabla \cdot \vec{E} = \dfrac{\rho}{\varepsilon_0} 
\end{equation*}

\begin{equation*}
  \oiint_{A} \vec{B} \cdot \md \vec{S} = 0
  \quad \Rightarrow \quad
  \nabla \cdot \vec{B} = 0
\end{equation*}

Ampere's Circuital Law

\begin{equation*}
  \oint_{C} \vec{B} \cdot \md \vec{l} = \mu_{0} \iint_{A} \left( \vec{J} + \varepsilon_{0} \dfrac{\partial \vec{E}}{\partial t}  \right) \cdot \md \vec{S}
  \quad \Rightarrow \quad
  \nabla \times \vec{B} = \mu_0 \varepsilon_0 \cdot \dfrac{\partial \vec{E}}{\partial t} 
\end{equation*}

We can now take the derivatives of the 4 equations

\begin{equation*}
  \left\{
  \begin{aligned}
    \nabla \times \vec{E} &= - \dfrac{\partial B}{\partial t} \\
    \nabla \times \vec{B} &= \mu_{0} \varepsilon_{0} \cdot \dfrac{\partial \vec{E}}{\partial t}  \\
    \nabla \cdot \vec{E} &= \dfrac{\rho}{\varepsilon_{0}} \\
    \nabla \cdot \vec{B} &= 0
  \end{aligned}
  \right.
  \quad \Rightarrow \quad
  \left\{
  \begin{aligned}
    \nabla^{2} \vec{E} = \mu_{0} \varepsilon_{0} \cdot \dfrac{\partial^{2} \vec{E}}{\partial t^{2}}= \dfrac{1}{c^2} \cdot \dfrac{\partial^2 \vec{E}}{\partial t^2}  \\
    \nabla^{2} \vec{B} = \mu_{0} \varepsilon_{0} \cdot \dfrac{\partial^{2} \vec{B}}{\partial t^{2}}  = \dfrac{1}{c^2} \cdot \dfrac{\partial^2 \vec{B}}{\partial t^2}  
  \end{aligned}
  \right.
\end{equation*}

Which indicates the speed of electromagnetic wave is exactly the speed of light.

Furthermore, it can be seen that the electric field and magnetic field are transverse. They are perpendicular to each other. We assume the electric field is parallel to the y-axis.

\begin{equation*}
  \begin{aligned}
    E_{y} (x,t) = E_{0} \cos \left[ \omega \left( t - x/c  \right) + \varepsilon \right]
  \end{aligned}
\end{equation*}

According to Faraday's Law

\begin{equation*}
  \begin{aligned}
    \dfrac{\partial E_{y}}{\partial x} = - \dfrac{\partial B_{z}}{\partial t}  
  \end{aligned}
\end{equation*}

We can calculate $B_{z}$

\begin{equation*}
  \begin{aligned}
    B_z = \dfrac{1}{c} \cdot E_{0} \cos \left[ \omega \left( t - x/c  \right) + \varepsilon \right]
  \end{aligned}
\end{equation*}

So that

\begin{equation*}
  \begin{aligned}
    E_y=v B_z =
  \end{aligned}
  \left\{
  \begin{aligned}
    & \dfrac{1}{\sqrt{\mu_0 \varepsilon_0}} \cdot B_z && \quad \text{in vacuum} \\
    & \dfrac{1}{\sqrt{\mu \varepsilon}} \cdot B_z && \quad \text{not in vacuum}
  \end{aligned}
  \right.
\end{equation*}

\section{Energy}

\begin{equation*}
  \begin{aligned}
    u_{E} = \dfrac{1}{2} \cdot \dfrac{\varepsilon_0}{1} \cdot E^{2} \quad\quad 
    u_{B} = \dfrac{1}{2} \cdot \dfrac{1}{\mu_0} \cdot  B^{2} \quad\quad 
    u_{E} = u_{B} \quad\quad 
    u = u_{E} + u_{B} = \varepsilon_{0} E^{2} = \dfrac{1}{\mu_{0}} B^{2}
  \end{aligned}
\end{equation*}

\begin{equation*}
  \begin{aligned}
    S = uc = \varepsilon_0 c E^2 \quad\quad \text{(Power: Transport of Energy per unit time across a unit area)}
  \end{aligned}
\end{equation*}

\begin{equation*}
  \begin{aligned}
    \vec{S} = \dfrac{1}{\mu} \cdot \vec{E} \times \vec{B} = c^2 \varepsilon \cdot \vec{E} \times \vec{B} \quad\quad \text{(Poynting Vector)}
  \end{aligned}
  \quad\quad 
  \begin{aligned}
    I = \dfrac{S}{2} = \dfrac{\varepsilon_0 c}{2} E_0^2 \quad\quad \text{(Irradiance)}
  \end{aligned}
\end{equation*}

\section{Radiation Pressure}

\begin{equation*}
  \begin{aligned}
    P(t) = u = u_E + u_B = \dfrac{S}{c} \quad\quad \text{Radiation Pressure equals energy density of the EM wave} 
  \end{aligned}
\end{equation*}

\begin{equation*}
  \begin{aligned}
    \left< P(t) \right>_T = \dfrac{1}{2} \cdot \dfrac{S}{c} = \dfrac{I}{c} \quad\quad \text{Average Radiation Pressure} 
  \end{aligned}
\end{equation*}

\begin{equation*}
  \begin{aligned}
    AP = \dfrac{\Delta p}{\Delta t}
    \quad \Rightarrow \quad
    Ac \Delta t P = c \Delta p
    \quad \Rightarrow \quad 
    p_V = \dfrac{P}{c} = \dfrac{S}{c^2} \quad\quad \text{Momentum per Volume} 
  \end{aligned}
\end{equation*}

\section{Light in Bulk Matter}

\subsection{Speed of light and Dielectric Constant}

\begin{equation*}
  \begin{aligned}
    v = \dfrac{1}{\sqrt{\varepsilon \mu}} 
    \quad\quad 
    n = \dfrac{c}{v} = \sqrt{\dfrac{\varepsilon \mu}{\varepsilon_0 \mu_0} } = \sqrt{\dfrac{\varepsilon}{\varepsilon_0} }
  \end{aligned}
\end{equation*}

\subsection{Dispersion}

For gas and solid

\begin{equation*}
  \begin{aligned}
    m_e \dfrac{\md ^2 x}{\md t^2} + \gamma m_e \dfrac{\md x}{\md t} + m_e \omega_0^2 x = - e E \left( t \right)
  \end{aligned}
\end{equation*}
\begin{equation*}
  \begin{aligned}
    E \left( t \right) = E_0 \exp \left( - i \omega t \right)
  \end{aligned}
\end{equation*}
Assume
\begin{equation*}
  \begin{aligned}
    x = x_0 \exp \left( - i \omega t \right)
  \end{aligned}
\end{equation*}
We got a solution
\begin{equation*}
  \begin{aligned}
    x_0 \left( \omega_0^2 - \omega^2 - i \gamma \omega \right) = - \dfrac{e E_0}{m_e} 
  \end{aligned}
\end{equation*}
\begin{equation*}
  \begin{aligned}
    x_0 = - \dfrac{e E_0}{m_e \left( \omega_0^2 - \omega^2 - i \gamma \omega \right)} 
  \end{aligned}
\end{equation*}
\begin{equation*}
  \begin{aligned}
    x \left( t \right) = - \dfrac{e E \left( t \right)}{m_e \left( \omega_0^2 - \omega^2 - i \gamma \omega \right)}
  \end{aligned}
\end{equation*}
\begin{equation*}
  \begin{aligned}
    P \left( t \right) = - N e x \left( t \right) = \dfrac{N e^2 E \left( t \right)}{m_e \left( \omega_0^2 - \omega^2 - i \gamma \omega \right)} 
  \end{aligned}
\end{equation*}
\begin{equation*}
  \begin{aligned}
    \varepsilon_r = \dfrac{\varepsilon}{\varepsilon_0} = n^2 = 1 + \dfrac{P}{\varepsilon_0 E} = 1 + \dfrac{N e^2}{\varepsilon_0 m_e \left( \omega_0^2 - \omega^2 - i \gamma \omega \right)}  
  \end{aligned}
\end{equation*}
\begin{equation*}
  \left\{
  \begin{aligned}
    \mathrm{Re} \left( \varepsilon_{r} \right) &= 1 + \dfrac{N e^2 \left( \omega_0^2 - \omega^2 \right)}{\varepsilon_0 m_e \left[ \left( \omega_0^2 - \omega^2 \right)^2 + \gamma^2 \omega^2 \right]}  \\
    \mathrm{Im} \left( \varepsilon_{r} \right) &= \dfrac{N e^2 \gamma \omega}{\varepsilon_0 m_e \left[ \left( \omega_0^2 - \omega^2 \right)^2 + \gamma^2 \omega^2 \right]}  
  \end{aligned}
  \right.
\end{equation*}
When $\gamma = 0$
\begin{equation*}
  \begin{aligned}
    \varepsilon_r = n^2 = 1 + \dfrac{N e^2}{\varepsilon_0 m \left( \omega_0^2 - \omega^2 \right)} 
  \end{aligned}
\end{equation*}

For metal

\begin{equation*}
  \begin{aligned}
    m_e \dfrac{\md ^2 x}{\md t^2} + \gamma m_e \dfrac{\md x}{\md t} = - e E \left( t \right)
  \end{aligned}
\end{equation*}

\begin{equation*}
  \begin{aligned}
    \varepsilon_r = 1 - \dfrac{N e^2}{\varepsilon_0 m_e \left(\omega^2 + i \gamma \omega \right)} = 1 - \dfrac{\omega_p^2}{\omega \left( \omega + i \gamma \right)}  
  \end{aligned}
\end{equation*}

%%% Local Variables:
%%% mode: latex
%%% TeX-master: "Optics"
%%% End:
