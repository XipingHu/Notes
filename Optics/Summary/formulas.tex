\documentclass{article}

% Chinese Support using xeCJK
% \usepackage{xeCJK}
% \setCJKmainfont{SimSun}

% Chinese Support using CTeX
\usepackage{ctex}

% Math Support
\usepackage{amsmath}
\usepackage{amsfonts}
\usepackage{amssymb}
\usepackage{wasysym}
\newcommand{\angstrom}{\text{\normalfont\AA}}

% Graphics Support
\usepackage{graphicx}
\usepackage{float}

% Reduced page margin
\usepackage{geometry}
\geometry{a4paper,scale=0.8}

\usepackage{caption}
\usepackage{subcaption}

% d and e should be math operators
\newcommand*{\dif}{\mathop{}\!\mathrm{d}}
\newcommand*{\md}{\mathop{}\!\mathrm{d}}
\newcommand*{\me}{\mathrm{e}}

% No indent for each paragraph
\usepackage{parskip}
\setlength{\parindent}{0cm}

% Bold style for Greek letters
\usepackage{bm}
\let\Oldmathbf\mathbf
\renewcommand{\mathbf}[1]{\boldsymbol{\Oldmathbf{#1}}}

% More space for dfrac in cell
\usepackage{cellspace}
\setlength{\cellspacetoplimit}{5pt}
\setlength{\cellspacebottomlimit}{5pt}

% SI units
\newcommand{\si}[1]{\  \mathrm{#1}}

\usepackage{multicol}

% Multi-line author information
\usepackage{authblk}
\author{胡喜平}
\affil{https://hxp.plus/}

\title{光学要背的公式}

\begin{document}

\maketitle

\begin{multicols}{2}

\section{光的波动理论}

电场和磁场间的数量关系 $E = c B$
\\\\
光速和介电常数 $c = \dfrac{1}{\sqrt{\varepsilon_0 \mu_0}} $
\\\\
玻印亭矢量 $\vec{S} = \vec{E} \times \vec{H} = \dfrac{1}{\mu_0} \vec{E} \times \vec{B} $
\\\\
光强 $I = \dfrac{1}{2} c^2 \varepsilon_0 E^2 $
\\\\
向$x$正方向传播的波 $y = A \cos \left( k x - \omega t \right)$
\\\\
相速度 $v = \dfrac{\omega}{k} $
\\\\
群速度 $v_p = \dfrac{\partial \omega}{\partial k} $

\section{光的传播}

$r_{\parallel} = \dfrac{n_t \cos \theta_i - n_i \cos \theta_t}{n_t \cos \theta_i + n_i \cos \theta_t} $
\\\\
$r_{\perp} = \dfrac{n_i \cos \theta_i - n_t \cos \theta_t}{n_i \cos \theta_i + n_t \cos \theta_t} $
\\\\
$t_{\parallel} = \dfrac{2 n_i \cos \theta_i}{n_t \cos \theta_i + n_i \cos \theta_t} $
\\\\
$t_{\perp} = \dfrac{2 n_i \cos \theta_i }{n_i \cos \theta_i + n_t \cos \theta_t} $
\\\\
$R = r^2$
\\\\
$T = \dfrac{n_t \cos \theta_t}{n_i \cos \theta_i} t^2 $
\\\\
隐逝场 设$\vec{E} = E_0 \exp \left[ i \left( \vec{k} \cdot \vec{r} - \omega t \right) \right]$
\\\\
布儒斯特角 $\tan \theta_p = \dfrac{n_t}{n_i} $
\\\\
牛顿公式 $x_o x_i = f$

\section{几何光学}

球面镜 $\dfrac{n_1}{s_o} + \dfrac{n_2}{s_i} = \dfrac{n_2 - n_1}{R}  $
\\\\
透镜成像公式
\\\\
$\dfrac{n_m}{s_{o1}} + \dfrac{n_m}{s_{i2}} = \left( n_l - n_m \right) \left( \dfrac{1}{R_1} - \dfrac{1}{R_2}   \right) + \dfrac{n_l d}{\left( s_{i1} - d \right)s_{i1}}   $
\\\\
面镜焦距 $f = - \dfrac{R}{2} $
\\\\
放大率 $M_T = - \dfrac{s_i}{s_o} $
\\\\
棱镜最小偏向角 $n = \dfrac{\sin \left[ \left( \delta_m + \alpha \right) \right] / 2}{\sin \left( \alpha / 2 \right)} $
\\\\
矩阵光学 $R = \left[
  \begin{array}{ccc}
   1 & - \Phi \\
   0 & 1 \\
  \end{array}
\right ]
\quad T = \left[
  \begin{array}{ccc}
   1 & 0 \\
   d/n & 1 \\
  \end{array}
\right ]
$

\section{光的偏振}

$E_y$ 比 $E_x$ 领先 $\pi/2$ 是右旋
\\\\
石英正单轴 \ 方解石负单轴
\\\\
平行光轴折射率 $n_e$
\\\\
e光方程 $\dfrac{k_{\parallel}^2}{n_o^2} + \dfrac{k_{prep}^2}{n_e^2} = k_0^2  $
\\\\
琼斯矢量 $\left[
  \begin{array}{cc}
   E_{0x} \\
   E_{0y} \exp \left( i \theta \right)  \\
  \end{array}
\right ]
$ ($x$ 比 $y$ 领先 $\delta$)

\section{光的干涉}

法布里珀罗干涉仪 $\delta = \dfrac{4 \pi}{\lambda} h \cos \theta + 2 \phi $

\section{光的衍射}

夫琅禾费衍射亮纹 $\sin \theta_i = \dfrac{k \lambda}{d} $
\\\\
夫琅禾费衍射缺级 $\sin \theta_i = \dfrac{k \lambda}{a} $

\end{multicols}

\end{document}