\documentclass{article}

% Chinese Support using xeCJK
% \usepackage{xeCJK}
% \setCJKmainfont{SimSun}

% Chinese Support using CTeX
\usepackage{ctex}

% Math Support
\usepackage{amsmath}
\usepackage{amsfonts}
\usepackage{amssymb}
\usepackage{wasysym}
\newcommand{\angstrom}{\text{\normalfont\AA}}

\usepackage{fancyhdr}

% Graphics Support
\usepackage{graphicx}
\usepackage{float}

% Reduced page margin
\usepackage{geometry}
\geometry{a4paper,scale=0.8}

\usepackage{caption}
\usepackage{subcaption}

% d and e should be math operators
\newcommand*{\dif}{\mathop{}\!\mathrm{d}}
\newcommand*{\md}{\mathop{}\!\mathrm{d}}
\newcommand*{\me}{\mathrm{e}}
\newcommand*{\mh}{\mathrm{h}}

% No indent for each paragraph
% \usepackage{parskip}
% \setlength{\parindent}{0cm}

% Bold style for Greek letters
\usepackage{bm}
\let\Oldmathbf\mathbf
\renewcommand{\mathbf}[1]{\boldsymbol{\Oldmathbf{#1}}}

% More space for dfrac in cell
\usepackage{cellspace}
\setlength{\cellspacetoplimit}{5pt}
\setlength{\cellspacebottomlimit}{5pt}

% SI units
\newcommand{\si}[1]{\  \mathrm{#1}}

% Multi-line author information
\usepackage{authblk}
\author{物理(4+4)1801 \quad 胡喜平 \quad 学号 U201811966}

\affil{网站 https://hxp.plus/ \quad 邮件 hxp201406@gmail.com}

\title{《量子力学教程》课后习题——第一章\ 绪论}

\pagestyle{fancy}
\fancyhf{}
\lhead{源码地址:https://github.com/hxp-plus/Notes/blob/master/Qumntum-Mechenics/Homework/}
\rfoot{第 \thepage 页}
\renewcommand{\headrulewidth}{1pt}
\renewcommand{\footrulewidth}{1pt}

\begin{document}

\maketitle\thispagestyle{fancy}

\paragraph{1.1}

黑体辐射公式为

\begin{equation*}
  \begin{aligned}
    \rho_{\nu} \md \nu = \dfrac{8 \pi h \nu^3}{c^3} \cdot \dfrac{1}{\exp \left[\dfrac{h \nu}{k_B T} \right] - 1} \md \nu  
  \end{aligned}
\end{equation*}

其中 $\rho_{\nu} \md \nu$ 是频率在 $\nu$ 到 $\nu + \md \nu$ 之内辐射能量密度。频率 $\nu$ 和波长 $\lambda$ 的关系为

\begin{equation*}
  \begin{aligned}
    \nu = \dfrac{c}{\lambda} 
  \end{aligned}
\end{equation*}

因此

\begin{equation*}
  \begin{aligned}
    \md \nu = - \dfrac{c}{\lambda^2} \md \lambda 
  \end{aligned}
\end{equation*}

设波长在 $\lambda$ 到 $\lambda + \md \lambda$ 之间的辐射能量密度为 $\rho_{\lambda}$,则

\begin{equation*}
  \begin{aligned}
    \rho_{\lambda} \md \lambda = \dfrac{8 \pi h}{c^3} \cdot \dfrac{c^3 / \lambda ^3}{\exp \left[ \dfrac{h c}{k_B} \cdot \dfrac{1}{\lambda T}   \right] - 1} \cdot \left( \dfrac{c}{\lambda^2} \md \lambda  \right)
    = 8 \pi h c \cdot \dfrac{\lambda^{-5}}{\exp \left[ \dfrac{h c}{k_B} \cdot \dfrac{1}{\lambda T}   \right] - 1 } \md \lambda 
  \end{aligned}
\end{equation*}

求极大值需要 $\dfrac{\md \rho_{\lambda}}{\md \lambda} = 0 $,即

\begin{equation*}
  \begin{aligned}
    \left\{ \exp \left[ \dfrac{hc}{k_B} \cdot \dfrac{1}{\lambda T}   \right] - 1 \right\} \left( - 5 \lambda^{-6} \right) - \lambda^{-5} \cdot \exp \left[ \dfrac{hc}{k_B} \cdot \dfrac{1}{\lambda T}   \right] \cdot \dfrac{hc}{k_B} \cdot \dfrac{1}{T} \cdot \left( - \dfrac{1}{\lambda^2}  \right) = 0
  \end{aligned}
\end{equation*}

化简后得到

\begin{equation*}
  \begin{aligned}
    \left\{ 5 - \dfrac{hc}{k_B} \cdot \dfrac{1}{\lambda T}   \right\} \cdot \exp \left[ \dfrac{hc}{k_B} \cdot \dfrac{1}{\lambda T}   \right] = 5
  \end{aligned}
\end{equation*}

令 $\alpha = \dfrac{hc}{k_B} \cdot \dfrac{1}{\lambda T}  $,化简得到

\begin{equation*}
  \begin{aligned}
    5 + \left( - \alpha \right) = 5 \me^{-\alpha}
  \end{aligned}
\end{equation*}

用计算器解得 $\alpha = 4.96$,即

\begin{equation*}
  \begin{aligned}
    \dfrac{hc}{k_B} \cdot \dfrac{1}{\lambda T} = 4.96  
  \end{aligned}
\end{equation*}

则

\begin{equation*}
  \begin{aligned}
    \lambda T = \dfrac{h c}{k_B \alpha} = 2.9 \times 10^3 \si{m \cdot K} 
  \end{aligned}
\end{equation*}

\paragraph{1.2}

电子的德布罗意波长为

\begin{equation*}
  \begin{aligned}
    \lambda = \dfrac{h}{\sqrt{2 m E}} = 0.708 \si{nm} 
  \end{aligned}
\end{equation*}

\paragraph{1.3}

氦原子由两个质子两个中子组成,它的质量为

\begin{equation*}
  \begin{aligned}
    m = 2 m_p + 2 m_n
  \end{aligned}
\end{equation*}

氢原子的德布罗意波长为

\begin{equation*}
  \begin{aligned}
    \lambda = \dfrac{h}{\sqrt{2 m E}} = \dfrac{h}{\sqrt{6 \left( m_p + m_n \right) k_B T}} = 1.26 \si{nm}  
  \end{aligned}
\end{equation*}

\paragraph{1.4 (1)}

波尔-索末菲量子化条件为

\begin{equation*}
  \begin{aligned}
    \oint m \dfrac{\md x}{\md t} \md x = \oint m \left( \dfrac{\md x}{\md t}  \right)^2 \md t = \left( n + \dfrac{1}{2}  \right) \mh 
  \end{aligned}
\end{equation*}

即

\begin{equation*}
  \begin{aligned}
    \oint E - E_p \md t = \oint E_k \md t = \oint \dfrac{1}{2} m \left( \dfrac{\md x}{\md t}  \right)^2 \md t = \left( 2 n + 1  \right) \mh  
  \end{aligned}
\end{equation*}

其中$E$是简谐振子总能量,$E_k$是动能,$E_p$是势能,设简谐运动

\begin{equation*}
  \begin{aligned}
    m \left( \dfrac{\md x}{\md t}  \right)^2 = k x
  \end{aligned}
\end{equation*}

则

\begin{equation*}
  \begin{aligned}
    x = A \cos \left( \omega t \right)
  \end{aligned}
\end{equation*}

其中

\begin{equation*}
  \begin{aligned}
    \omega^2 = \dfrac{k}{m}
  \end{aligned}
\end{equation*}

量子化条件可以展开为

\begin{equation*}
  \begin{aligned}
    \oint E \md t - \oint \dfrac{1}{2} k x^2 \md t = \dfrac{2 \pi E}{\omega} - \oint \dfrac{1}{2} k \cdot A^2 \cos^2 \left( \omega t \right) \md t = \dfrac{2 \pi E}{\omega} - \dfrac{\pi}{\omega} \cdot \dfrac{1}{2}  k A^2 = \dfrac{2 \pi}{\omega} E - \dfrac{\pi}{\omega} E = \dfrac{\pi}{\omega} E   
  \end{aligned}
\end{equation*}

因此

\begin{equation*}
  \begin{aligned}
    \dfrac{\pi}{\omega} E = \left( 2 n + 1 \right) \mh 
  \end{aligned}
\end{equation*}

简谐振子能量表达式

\begin{equation*}
  \begin{aligned}
    E = \left( 2n + 1 \right) \mh \cdot \dfrac{\omega}{\pi} = \left( n + \dfrac{1}{2}  \right) \hbar \omega 
  \end{aligned}
\end{equation*}

\paragraph{1.5}

光子波长最大的情况下能量最小,恰好等于电子能量

\begin{equation*}
  \begin{aligned}
    \dfrac{h c}{\lambda_{max}} = m_e c^2 
  \end{aligned}
\end{equation*}

计算得出光子最大波长

\begin{equation*}
  \begin{aligned}
    \lambda_{max} = \dfrac{h}{m_e c} = 2.243 \times 10^{-12} \si{m} 
  \end{aligned}
\end{equation*}

\end{document} 
