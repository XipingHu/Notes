\documentclass{article}

% Chinese Support using xeCJK
% \usepackage{xeCJK}
% \setCJKmainfont{SimSun}

% Chinese Support using CTeX
\usepackage{ctex}

% Math Support
\usepackage{amsmath}
\usepackage{amsfonts}
\usepackage{amssymb}
\usepackage{wasysym}
\newcommand{\angstrom}{\text{\normalfont\AA}}

\usepackage{fancyhdr}

% Graphics Support
\usepackage{graphicx}
\usepackage{float}

% Reduced page margin
\usepackage{geometry}
\geometry{a4paper,scale=0.8}

\usepackage{caption}
\usepackage{subcaption}

% d and e should be math operators
\newcommand*{\dif}{\mathop{}\!\mathrm{d}}
\newcommand*{\md}{\mathop{}\!\mathrm{d}}
\newcommand*{\me}{\mathrm{e}}
\newcommand*{\mh}{\mathrm{h}}
\newcommand*{\Jmath}{J}

% No indent for each paragraph
\usepackage{parskip}
\setlength{\parindent}{0cm}

% Bold style for Greek letters
\usepackage{bm}
\let\Oldmathbf\mathbf
\renewcommand{\mathbf}[1]{\boldsymbol{\Oldmathbf{#1}}}

% More space for dfrac in cell
\usepackage{cellspace}
\setlength{\cellspacetoplimit}{5pt}
\setlength{\cellspacebottomlimit}{5pt}

% SI units
\newcommand{\si}[1]{\  \mathrm{#1}}

% Multi-line author information
\usepackage{authblk}
\author{物理(4+4)1801 \quad 胡喜平 \quad 学号 U201811966}

\affil{网站 https://hxp.plus/ \quad 邮件 hxp201406@gmail.com}

\title{《电动力学》课后习题——第二章\ 静电场}

\pagestyle{fancy}
\fancyhf{}
\lhead{源码地址:https://github.com/hxp-plus/Notes/blob/master/Electro-Dynamics/Homework/}
\rfoot{第 \thepage 页}
\renewcommand{\headrulewidth}{1pt}
\renewcommand{\footrulewidth}{1pt}

\begin{document}

\maketitle\thispagestyle{fancy}

\paragraph{2.1}

半径为$R$的电介质球,极化强度为$\vec{P}=K \dfrac{\vec{r}}{r^2} $,电容率为$\epsilon$

(1)计算束缚电荷的体密度和面密度

(2)计算自由电荷体密度

(3)计算球外和球内的电势

(4)求该带电介质球产生的静电场总能量

\paragraph{解}

计算束缚电荷密度,只需要对极化强度求梯度

\begin{equation*}
  \begin{aligned}
    \rho_p = - \nabla \cdot \vec{P} = \dfrac{K}{r^2} 
  \end{aligned}
\end{equation*}

面电荷密度用电场的边值关系求解

\begin{equation*}
  \begin{aligned}
    \sigma_p = - \vec{e}_n \cdot \left( 0 - \vec{P} \right) = \dfrac{k}{r} 
  \end{aligned}
\end{equation*}

在自由电荷密度为

\begin{equation*}
  \begin{aligned}
    \rho_f &= - \nabla \cdot \vec{D} = - \nabla \cdot \left( \varepsilon_0 \vec{E} + \vec{P} \right)
    = - \nabla \cdot \left( \dfrac{\varepsilon_0}{\varepsilon_0 \chi_e} + 1   \right) \vec{P}
    = - \nabla \cdot \left( \dfrac{\varepsilon_0}{\varepsilon_0 \left( \varepsilon_r - 1 \right) } +1 \right) \vec{P} \\
    &= - \nabla \cdot \left( \dfrac{\varepsilon_0}{\varepsilon - \varepsilon_0} + 1 \right) \vec{P}
    = - \nabla \cdot \dfrac{\varepsilon}{\varepsilon - \varepsilon_0} \vec{P} = \dfrac{\varepsilon}{\varepsilon - \varepsilon_0} \left( - \nabla \vec{P} \right)
    = \dfrac{\varepsilon}{\varepsilon - \varepsilon_0} \dfrac{K}{r^2}
  \end{aligned}
\end{equation*}

球内的电场为

\begin{equation*}
  \begin{aligned}
    \vec{E}_i = \dfrac{\vec{P}}{\varepsilon_0 \chi_e} = \dfrac{\vec{P}}{\varepsilon_0 \left( \varepsilon_r - 1 \right)} = \dfrac{\vec{P}}{\varepsilon - \varepsilon_0} 
    = \dfrac{K}{\varepsilon - \varepsilon_0} \cdot \dfrac{\vec{r}}{r^2}  
  \end{aligned}
\end{equation*}

球外电场为

\begin{equation*}
  \begin{aligned}
    \vec{E}_o = \dfrac{\int_0^R 4 \pi r^2 \rho_f \md r }{4 \pi \varepsilon_0 r^3} \vec{r}
    = \dfrac{\int_0^R r^2  \dfrac{\varepsilon}{\varepsilon - \varepsilon_0} \dfrac{K}{r^2} \md r }{\varepsilon_0 r^3} \vec{r}
    = \int_0^R   \dfrac{\varepsilon K }{\varepsilon_0 \left( \varepsilon - \varepsilon_0 \right) } \md r \cdot \dfrac{\vec{r}}{r^3} 
    = \dfrac{\varepsilon K R}{\varepsilon_0 \left( \varepsilon - \varepsilon_0 \right)} \cdot \dfrac{\vec{r}}{r^3} 
  \end{aligned}
\end{equation*}

球外电势为

\begin{equation*}
  \begin{aligned}
    \varphi_o = \int_r^{\infty} \vec{E}_o \md \vec{r} = \dfrac{\varepsilon K R}{\varepsilon_0 \left( \varepsilon - \varepsilon_0 \right)} \dfrac{1}{r}  
  \end{aligned}
\end{equation*}

球内电势为

\begin{equation*}
  \begin{aligned}
    \varphi_i = \int_R^{\infty} \vec{E}_o \md \vec{r} + \int_r^R \vec{E}_i \md \vec{r} = \dfrac{K}{\varepsilon - \varepsilon_0} \left[ \ln \dfrac{R}{r} + \dfrac{\varepsilon}{\varepsilon_0}   \right] 
  \end{aligned}
\end{equation*}

总能量为

\begin{equation*}
  \begin{aligned}
    W = \dfrac{1}{2} \int_0^R 4 \pi r^2 \varepsilon \vec{E}_i^2 \md r
    + \dfrac{1}{2} \int_R^{\infty} 4 \pi r^2 \varepsilon_0 \vec{E}_o^2 \md r
    = 2 \pi \varepsilon R \left( \dfrac{K}{\varepsilon - \varepsilon_0}  \right)^2
    \left( 1 + \dfrac{\varepsilon}{\varepsilon_0}  \right)
  \end{aligned}
\end{equation*}

\paragraph{2.2}

在均匀外电场中置入半径为$R_0$的导体球,试用分离变量法求下列两种情况的电势:

(1)导体球上接有电池,使球与地保持电势差$\Phi_0$

(2)导体球上带总电荷$Q$

\paragraph{解}

导体边界电势为$\Phi_0$时,设解为

\begin{equation*}
  \begin{aligned}
    \varphi =
    & \sum_n \left( a_n R^n + \dfrac{b_n}{R_n^{n+1}} \right) P_n \left( \cos \theta \right)
  \end{aligned}
\end{equation*}

边界条件为

\begin{equation*}
  \begin{aligned}
    & \left. \varphi \right|_{r= \infty} = - E_0 R \cos \theta \\
    & \left. \varphi \right|_{r=R_0} = \Phi_0
  \end{aligned}
\end{equation*}

代入边界条件得出

\begin{equation*}
  \begin{aligned}
    &a_1= - E_0 && \\
    &a_n = 0 && n \neq 1 \\
    &b_0 = \Phi_0 R_0 &&\\
    &b_1 = E_0 R_0^3 &&\\
    &b_n = 0 && n \neq 0,1
  \end{aligned}
\end{equation*}

所以

\begin{equation*}
  \begin{aligned}
    \varphi = \Phi_0 R_0 - E_0 \left( R - \dfrac{R_0^3}{R^2}  \right) \cos \theta
  \end{aligned}
\end{equation*}

当没有外接电池而是在导体球上放置电荷$Q$时,球面电势为

\begin{equation*}
  \begin{aligned}
    \Phi_0' = \dfrac{Q}{4 \pi \varepsilon_0} 
  \end{aligned}
\end{equation*}

方程和边界条件相同,因此

\begin{equation*}
  \begin{aligned}
    \varphi = \Phi_0' R_0 - E_0 \left( R - \dfrac{R_0^3}{R^2}  \right) \cos \theta
    = \dfrac{Q}{4 \pi \varepsilon_0} R_0 - E_0 \left( R - \dfrac{R_0^3}{R^2}  \right) \cos \theta
  \end{aligned}
\end{equation*}

\paragraph{2.4}

均匀介质球(电容率为$\varepsilon_1$)的中心置一自由电偶极子$\vec{p}_f$,球外充满了另一种介质(电容率为$\varepsilon_2$),求空间各点电势和极化电荷分布。

提示:$\varphi = \dfrac{\vec{p}_f \cdot \vec{R}}{ 4 \pi \varepsilon_1 R^3} + \varphi' $,而$\varphi'$满足拉普拉斯方程

\paragraph{解}

电偶极子产生的电势为

\begin{equation*}
  \begin{aligned}
    \varphi = \dfrac{\vec{p}_f \cdot \vec{R}}{4 \pi \varepsilon_0 R^3} 
  \end{aligned}
\end{equation*}

设

\begin{equation*}
  \begin{aligned}
    \varphi_i &= \dfrac{p_f\cos \theta}{4\pi \varepsilon_1 R^2} + \sum_n a_n R^n P_n \left( \cos \theta \right) && R < R_0 \\
    \varphi_o &= \dfrac{p_f\cos \theta}{4\pi \varepsilon_2 R^2} + \sum_n \dfrac{b_n}{R^{n+1}}  P_n \left( \cos \theta \right) && R > R_0 
  \end{aligned}
\end{equation*}

因为边界上电势连续

\begin{equation*}
  \begin{aligned}
    \dfrac{p_f\cos \theta}{4\pi \varepsilon_1 R_0^2} + \sum_n a_n R_0^n P_n \left( \cos \theta \right)
    =
    \dfrac{p_f\cos \theta}{4\pi \varepsilon_2 R_0^2} + \sum_n \dfrac{b_n}{R_0^{n+1}}  P_n \left( \cos \theta \right)
  \end{aligned}
\end{equation*}

\begin{equation*}
  \begin{aligned}
    \varepsilon_1 \left\{ - \dfrac{2p_f\cos \theta}{4\pi \varepsilon_1 R_0^3} + \sum_n n a_n R_0^{n-1} P_n \left( \cos \theta \right) \right\}
    =
    \varepsilon_2 \left\{ - \dfrac{2p_f\cos \theta}{4\pi \varepsilon_2 R_0^3} - \sum_n \left( n + 1 \right) \dfrac{b_n}{R_0^{n+2}}  P_n \left( \cos \theta \right) \right\}
  \end{aligned}
\end{equation*}


当$n=1$时

\begin{equation*}
  \begin{aligned}
    \dfrac{p_f}{4\pi \varepsilon_1 R_0^2} + a_1 R_0 
    =
    \dfrac{p_f}{4\pi \varepsilon_2 R_0^2} + \dfrac{b_1}{R_0^2} 
  \end{aligned}
\end{equation*}

\begin{equation*}
  \begin{aligned}
    \varepsilon_1 \left\{- \dfrac{2p_f}{4\pi \varepsilon_1 R_0^3} + a_1 \right\}
    =
    \varepsilon_2 \left\{ - \dfrac{2p_f}{4\pi \varepsilon_2 R_0^3} - 2 \dfrac{b_1}{R_0^3} \right\}
    \Rightarrow
    \varepsilon_1 a_1
    =
    - \dfrac{2 \varepsilon_2 b_1}{R_0^3}
  \end{aligned}
\end{equation*}

解得

\begin{equation*}
  \begin{aligned}
    a_1 = \dfrac{\left( \varepsilon_1 - \varepsilon_2 \right) p_f}{2\pi \varepsilon_1 \left( \varepsilon_1 + 2 \varepsilon_2 \right) R_0^3} 
    \quad\quad
    b_1 = \dfrac{\left( \varepsilon_2 - \varepsilon_1 \right) p_f}{4\pi \varepsilon_2 \left( \varepsilon_1 + 2\varepsilon_2 \right)} 
  \end{aligned}
\end{equation*}

当$n \neq 1$时,$a_n = b_n = 0$,因此方程的解为

\begin{equation*}
  \begin{aligned}
    \varphi_i &= \dfrac{p_f \cos \theta}{4\pi \varepsilon_1 R^2} + \dfrac{\left( \varepsilon_1 - \varepsilon_2 \right) p_f \cos \theta}{2\pi \varepsilon_1 \left( \varepsilon_1 + 2 \varepsilon_2 \right) R_0^3} R  && R < R_0 \\
    \varphi_o &= \dfrac{p_f \cos \theta}{4\pi \varepsilon_2 R^2} + \dfrac{\left( \varepsilon_2 - \varepsilon_1 \right) p_f \cos \theta}{4\pi \varepsilon_2 \left( \varepsilon_1 + 2\varepsilon_2 \right)} \dfrac{1}{R^2}  && R > R_0 
  \end{aligned}
\end{equation*}

\paragraph{2.8}

半径为$R_0$的导体球外充满均匀绝缘介质$\varepsilon$,导体球接地,离球心$a$处($a>R_0$)置一点电荷$Q_f$,试用分离变量法求空间各点电势,证明所得结果与镜像法结果相同

\paragraph{解}

球内电势为零,设球外电势为

\begin{equation*}
  \begin{aligned}
    \varphi = \dfrac{Q_f}{4 \pi \varepsilon_0 r} + \sum_n \dfrac{b_n}{R^{n+1}} P_n \left( \cos \theta \right)  
  \end{aligned}
\end{equation*}

当$R<a$时,展开

\begin{equation*}
  \begin{aligned}
    \dfrac{1}{r} = \dfrac{1}{\sqrt{R^2 + a^2 - 2 R a \cos \theta}}
    = \dfrac{1}{a} \sum _n \left( \dfrac{R}{a}  \right)^n P_n \left( \cos \theta \right) 
  \end{aligned}
\end{equation*}

因为$R_0 < a$,代入得到

\begin{equation*}
  \begin{aligned}
    \varphi = \dfrac{Q_f}{4 \pi \varepsilon_0} \dfrac{1}{a} \sum _n \left( \dfrac{R}{a}  \right)^n P_n \left( \cos \theta \right) + \sum_n \dfrac{b_n}{R^{n+1}} P_n \left( \cos \theta \right)
    = \left[ \sum _n \dfrac{Q_f}{4 \pi \varepsilon_0} \dfrac{R^n}{a^{n+1}} + \dfrac{b_n}{R^{n+1}} \right] P_n \left( \cos \theta \right)
  \end{aligned}
\end{equation*}

当$R=R_0$时,$\varphi =0$

\begin{equation*}
  \begin{aligned}
    \dfrac{Q_f}{4 \pi \varepsilon_0} \dfrac{R_0^n}{a^{n+1}} + \dfrac{b_n}{R_0^{n+1}} = 0
    \Rightarrow
    b_n = - \dfrac{Q_f}{4 \pi \varepsilon_0} \dfrac{R_0^{2n+1}}{a^{n+1}}
  \end{aligned}
\end{equation*}

因此方程的解为

\begin{equation}
  \label{eq:a281}
  \begin{aligned}
    \varphi = \dfrac{Q_f}{4 \pi \varepsilon_0} \left[ \dfrac{1}{r}  - \sum_n \dfrac{R_0^{2n+1}}{a^{n+1}} \dfrac{1}{R^{n+1}} P_n \left( \cos \theta \right) \right]
  \end{aligned}
\end{equation}

用电像法时只需要在$b=\dfrac{R_0^2}{a} $处放置$q_f=-\dfrac{R_0}{a} Q_f $的电荷,空间各点电势为

\begin{equation}
  \label{eq:a282}
  \begin{aligned}
    \varphi = \dfrac{Q_f}{4\pi \varepsilon_0} \left[ \dfrac{1}{r} - \dfrac{R_0}{a} \dfrac{1}{r'}    \right] 
  \end{aligned}
\end{equation}

其中
\begin{equation*}
  \begin{aligned}
    \dfrac{1}{r'} = \dfrac{1}{\sqrt{R^2 + b^2 - 2 R b \cos \theta}}
  \end{aligned}
\end{equation*}

因为$b<R$,展开式为


\begin{equation*}
  \begin{aligned}
    \dfrac{1}{\sqrt{R^2 + b^2 - 2 R b \cos \theta}} =
    \dfrac{1}{R} \sum _n \left( \dfrac{b}{R}  \right)^n P_n \left( \cos \theta \right) 
    = \dfrac{1}{R} \sum _n \left( \dfrac{R_0^2}{aR}  \right)^n P_n \left( \cos \theta \right)
    = \sum_n \dfrac{R_0^{2n}}{a^n} \dfrac{1}{R^{n+1}} P_n \left( \cos \theta \right)
  \end{aligned}
\end{equation*}

代入式(\ref{eq:a282})得到

\begin{equation*}
  \begin{aligned}
    \varphi = \dfrac{Q_f}{4\pi \varepsilon_0} \left[ \dfrac{1}{r} - \dfrac{R_0}{a} \sum_n \dfrac{R_0^{2n}}{a^n} \dfrac{1}{R^{n+1}} P_n \left( \cos \theta \right)  \right]
  \end{aligned}
\end{equation*}

和分离变量法的结果式(\ref{eq:a281})相同

\paragraph{2.11}

在接地的导体平面上有一半径为$a$的半球凸部,半球的球心在导体平面上,点电荷$Q$位于系统的对称轴上,并与平面相距为$b$($b>a$),试用镜像法求空间电势

\paragraph{解}

在原点的下方距离为$a$的地方放置一个电量为$-Q$的电荷,距离为$\dfrac{a^2}{b} $的地方放置一个$+qa/b$的电荷。之后在原点上方$\dfrac{a^2}{b} $处放置一个$-qa/b$的电荷,电势为

\begin{equation*}
  \begin{aligned}
    \varphi = \dfrac{Q}{4\pi \varepsilon_0} & \left[ \dfrac{1}{\sqrt{x^2 + y^2 + \left( z-b \right)^2}} - \dfrac{1}{\sqrt{x^2 + y^2 + \left( z + b \right)^2}} \right. \\
     & \left. - \dfrac{a/b}{\sqrt{x^2 + y^2 + \left( z - a^2/b \right)^2}} + \dfrac{a/b}{\sqrt{x^2 + y^2 + \left( z + a^2/b \right)^2}}   \right]
  \end{aligned}
\end{equation*}

\paragraph{2.12}

有一点电荷$Q$位于两个相互垂直的接地导体平面所围成的直角空间内,它到两个平面的距离为$a$和$b$,求空间电势

\paragraph{解}

在$(0,-a,b)$、$(0,a,-b)$、$(0,-a,-b)$各放置三个虚拟电荷$-Q$、$-Q$、$+Q$,电势为

\begin{equation*}
  \begin{aligned}
    \varphi = \dfrac{Q}{4\pi \varepsilon_0} & \left\{
    \left[ \left( x-a \right)^2 + \left( y-b \right)^2 + z^2 \right]^{-1/2}
    + \left[ \left( x+a \right)^2 + \left( y+b \right)^2 + z^2 \right]^{-1/2}
    \right. \\
    &\left.
    - \left[ \left( x+a \right)^2 + \left( y-b \right)^2 + z^2 \right]^{-1/2}
    - \left[ \left( x-a \right)^2 + \left( y+b \right)^2 + z^2 \right]^{-1/2} 
    \right\}
  \end{aligned}
\end{equation*}

\paragraph{2.18}

一半径为$R_0$的球面,在球坐标$0<\theta< \dfrac{\pi}{2} $的半球面上电势为$\varphi_0$,在$\dfrac{\pi}{2} < \theta < \pi $的半球面上电势为$-\varphi_0$,球空间各点电势

提示:

\begin{equation*}
  \begin{aligned}
    \int_0^1 P_n \left( x \right) \md x = \left. \dfrac{P_{n+1} \left( x \right) - P_{n-1} \left( x \right)}{2n+1} \right|_0^1
  \end{aligned}
\end{equation*}

\begin{equation*}
  \begin{aligned}
    P_n \left( 1 \right) = 1
  \end{aligned}
\end{equation*}

\begin{equation*}
  \begin{aligned}
    P_n \left( 0 \right) =
  \end{aligned}
  \left\{
  \begin{aligned}
    &0 && n=2k+1 \\
    & \left( -1 \right)^{n/2} \dfrac{1 \cdot 3 \cdot 5 \cdots \left( n-1 \right)}{2 \cdot 4 \cdot 6 \cdots n} && n=2k 
  \end{aligned}
  \right.
\end{equation*}

\paragraph{解}

设电势为

\begin{equation*}
  \begin{aligned}
    \varphi_i &= \sum_n a_n R^n P \left( \cos \theta \right) && R<R_0 \\
    \varphi_o &= \sum_n \dfrac{b_n}{R^{n+1}} P_n \left( \cos \theta \right) && R>R_0 
  \end{aligned}
\end{equation*}

下面求解球内电势分布,将$\varphi_i$展开

\begin{equation*}
  \begin{aligned}
    \varphi_i \left( x \right) = \sum_n \dfrac{2n+1}{2} \int_{-1}^{+1} \varphi_i \left( x \right) P_n \left( x \right) \md x P \left( x \right) 
  \end{aligned}
\end{equation*}

比较系数得

\begin{equation*}
  \begin{aligned}
    \dfrac{2n+1}{2} \int_{-1}^{+1} \varphi_i \left( x \right) P_n \left( x \right) \md x
    = a_n R^n
  \end{aligned}
\end{equation*}

在$R=R_0$处

\begin{equation*}
  \begin{aligned}
    a_n R_0^n &=
    \dfrac{2n+1}{2} \left[ - \int_{-1}^{0} \varphi_0 \left( x \right) P_n \left( x \right) \md x + \int_{0}^{+1} \varphi_0 \left( x \right) P_n \left( x \right) \md x \right]
    = \left( 2n+1 \right) \varphi_0 \left( x \right) \int_0^1 P_n \left( x \right) \md x \\
    &= \left( 2n+1 \right) \varphi_0 \left( x \right) \left. \left[ \dfrac{P_{n+1} \left( x \right) - P_{n-1} \left( x \right)}{2n+1} \right] \right|_0^1 \\
    &= \varphi_0 \left( x \right) \left. \left[  P_{n+1} \left( x \right) - P_{n-1} \left( x \right) \right] \right|_0^1
  \end{aligned}
\end{equation*}

其中

\begin{equation*}
  \begin{aligned}
    \left. \left[  P_{n+1} \left( x \right) - P_{n-1} \left( x \right) \right] \right|_0^1
    &= P_{n+1} \left( 1 \right) - P_{n-1} \left( 1 \right) - P_{n+1} \left( x \right) + P_{n-1} \left( 0 \right) = P_{n-1} \left( 0 \right) - P_{n+1} \left( 0 \right) 
  \end{aligned}
\end{equation*}

当$n$为偶数时,$P_{n-1}=P_{n+1}=0$,$a_n=0$,当$n$为奇数时

\begin{equation*}
  \begin{aligned}
    P_{n-1} \left( 0 \right) &= \left( -1  \right)^{\left( n-1 \right)/ 2} \cdot \dfrac{1 \cdot 3 \cdot 5 \dots \left( n-2 \right)}{2 \cdot 4 \cdot 6 \dots \left( n-1 \right)}  \\
    P_{n+1} \left( 0 \right) &= \left( -1  \right)^{\left( n+1 \right)/ 2} \cdot \dfrac{1 \cdot 3 \cdot 5 \dots \left( n-2 \right)}{2 \cdot 4 \cdot 6 \dots \left( n-1 \right)} \cdot \dfrac{n}{n+1}   
  \end{aligned}
\end{equation*}

因此

\begin{equation*}
  \begin{aligned}
    \dfrac{a_n R_0^n }{\varphi_0}  &=
    P_{n-1} \left( 0 \right) - P_{n+1} \left( 0 \right) \\
    &= \left( -1  \right)^{\left( n-1 \right)/ 2} \cdot \dfrac{1 \cdot 3 \cdot 5 \dots \left( n-2 \right)}{2 \cdot 4 \cdot 6 \dots \left( n-1 \right)} \cdot \left[ 1 + \dfrac{n}{n+1}  \right] \\
    &= \left( -1  \right)^{\left( n-1 \right)/ 2} \cdot \dfrac{1 \cdot 3 \cdot 5 \dots \left( n-2 \right)}{2 \cdot 4 \cdot 6 \dots \left( n-1 \right)} \cdot \dfrac{2n+1}{n+1}
  \end{aligned}
\end{equation*}

球内电势为

\begin{equation*}
  \begin{aligned}
    \varphi_i &= \varphi_0 \sum_n \left[ \left( -1  \right)^{\left( n-1 \right)/ 2} \cdot \dfrac{1 \cdot 3 \cdot 5 \dots \left( n-2 \right)}{2 \cdot 4 \cdot 6 \dots \left( n-1 \right)} \cdot \dfrac{2n+1}{n+1}
 \right] \dfrac{R^n}{R_0^n}  P \left( \cos \theta \right) && R<R_0, n=2k+1
  \end{aligned}
\end{equation*}

下面求解球外电势分布,将$\varphi_o$展开

\begin{equation*}
  \begin{aligned}
    \varphi_o \left( x \right) = \sum_n \dfrac{2n+1}{2} \int_{-1}^{+1} \varphi_o \left( x \right) P_n \left( x \right) \md x P \left( x \right) 
  \end{aligned}
\end{equation*}

比较系数得

\begin{equation*}
  \begin{aligned}
    \dfrac{2n+1}{2} \int_{-1}^{+1} \varphi_o \left( x \right) P_n \left( x \right) \md x
    = \dfrac{b_n}{R^{n+1}} 
  \end{aligned}
\end{equation*}

在$R=R_0$处

\begin{equation*}
  \begin{aligned}
    \dfrac{b_n}{R_0^{n+1}}  &=
    \dfrac{2n+1}{2} \left[ - \int_{-1}^{0} \varphi_0 \left( x \right) P_n \left( x \right) \md x + \int_{0}^{+1} \varphi_0 \left( x \right) P_n \left( x \right) \md x \right]
    = \left( 2n+1 \right) \varphi_0 \left( x \right) \int_0^1 P_n \left( x \right) \md x \\
    &= \left( 2n+1 \right) \varphi_0 \left( x \right) \left. \left[ \dfrac{P_{n+1} \left( x \right) - P_{n-1} \left( x \right)}{2n+1} \right] \right|_0^1 \\
    &= \varphi_0 \left( x \right) \left. \left[  P_{n+1} \left( x \right) - P_{n-1} \left( x \right) \right] \right|_0^1
  \end{aligned}
\end{equation*}

其中

\begin{equation*}
  \begin{aligned}
    \left. \left[  P_{n+1} \left( x \right) - P_{n-1} \left( x \right) \right] \right|_0^1
    &= P_{n+1} \left( 1 \right) - P_{n-1} \left( 1 \right) - P_{n+1} \left( x \right) + P_{n-1} \left( 0 \right) = P_{n-1} \left( 0 \right) - P_{n+1} \left( 0 \right) 
  \end{aligned}
\end{equation*}

当$n$为偶数时,$P_{n-1}=P_{n+1}=0$,$a_n=0$,当$n$为奇数时

\begin{equation*}
  \begin{aligned}
    P_{n-1} \left( 0 \right) &= \left( -1  \right)^{\left( n-1 \right)/ 2} \cdot \dfrac{1 \cdot 3 \cdot 5 \dots \left( n-2 \right)}{2 \cdot 4 \cdot 6 \dots \left( n-1 \right)}  \\
    P_{n+1} \left( 0 \right) &= \left( -1  \right)^{\left( n+1 \right)/ 2} \cdot \dfrac{1 \cdot 3 \cdot 5 \dots \left( n-2 \right)}{2 \cdot 4 \cdot 6 \dots \left( n-1 \right)} \cdot \dfrac{n}{n+1}   
  \end{aligned}
\end{equation*}

因此

\begin{equation*}
  \begin{aligned}
    \dfrac{b_n}{R_0^{n+1} \varphi_0}  &=
    P_{n-1} \left( 0 \right) - P_{n+1} \left( 0 \right) \\
    &= \left( -1  \right)^{\left( n-1 \right)/ 2} \cdot \dfrac{1 \cdot 3 \cdot 5 \dots \left( n-2 \right)}{2 \cdot 4 \cdot 6 \dots \left( n-1 \right)} \cdot \left[ 1 + \dfrac{n}{n+1}  \right] \\
    &= \left( -1  \right)^{\left( n-1 \right)/ 2} \cdot \dfrac{1 \cdot 3 \cdot 5 \dots \left( n-2 \right)}{2 \cdot 4 \cdot 6 \dots \left( n-1 \right)} \cdot \dfrac{2n+1}{n+1}
  \end{aligned}
\end{equation*}

球外电势为

\begin{equation*}
  \begin{aligned}
    \varphi_o &= \varphi_0 \sum_n \left[ \left( -1  \right)^{\left( n-1 \right)/ 2} \cdot \dfrac{1 \cdot 3 \cdot 5 \dots \left( n-2 \right)}{2 \cdot 4 \cdot 6 \dots \left( n-1 \right)} \cdot \dfrac{2n+1}{n+1}
 \right] \dfrac{R_0^{n+1}}{R^{n+1}}  P \left( \cos \theta \right) && R>R_0, n=2k+1
  \end{aligned}
\end{equation*}

综上

\begin{equation*}
  \begin{aligned}
    \varphi = 
  \end{aligned}
  \left\{
  \begin{aligned}
    &\varphi_0 \sum_n \left[ \left( -1  \right)^{\left( n-1 \right)/ 2} \cdot \dfrac{1 \cdot 3 \cdot 5 \dots \left( n-2 \right)}{2 \cdot 4 \cdot 6 \dots \left( n-1 \right)} \cdot \dfrac{2n+1}{n+1}
 \right] \dfrac{R^n}{R_0^n}  P \left( \cos \theta \right) && R<R_0, n=2k+1 \\
    &\varphi_0 \sum_n \left[ \left( -1  \right)^{\left( n-1 \right)/ 2} \cdot \dfrac{1 \cdot 3 \cdot 5 \dots \left( n-2 \right)}{2 \cdot 4 \cdot 6 \dots \left( n-1 \right)} \cdot \dfrac{2n+1}{n+1}
 \right] \dfrac{R_0^{n+1}}{R^{n+1}}  P \left( \cos \theta \right) && R>R_0, n=2k+1
  \end{aligned}
  \right.
\end{equation*}

\paragraph{2.19}

上题能用格林函数解吗?结果如何?

\paragraph{解}

格林函数为

\begin{equation*}
  \begin{aligned}
    G \left( x', x \right) &= 
      \dfrac{1}{4\pi \varepsilon_0} \left\{ \left[ \left( R^2 - R'^2 - 2 R R' \cos \alpha \right) \right]^{-0.5} -
      \left[ \left( \dfrac{R R'}{R_0}  \right)^2 + R_0^2 - 2 RR' \cos \alpha \right]^{-0.5}
    \right\} \\
    &= \dfrac{1}{4\pi \varepsilon_0} \sum_n \left[ \dfrac{R^n}{R'^{n+1}} + \dfrac{\left( RR'/R_0 \right)^n}{R_0^{n+1}}   \right] P_n \left( x \right) P_n \left( x' \right) \\
    \dfrac{\partial G}{\partial R'} &=
    - \dfrac{1}{4\pi \varepsilon_0} \sum_n \left[ \dfrac{n+1}{R'^{n+2}} R^n + \dfrac{nR^n R'^{n-1}}{R_0^{2n+1}} \right] P_n \left( x \right) P_n \left( x' \right)
  \end{aligned}
\end{equation*}

\begin{equation*}
  \begin{aligned}
    \left. \dfrac{\partial G}{\partial R'} \right|_{R'=R_0} &=
    - \dfrac{1}{4\pi \varepsilon_0} \sum_n \left[ \dfrac{n+1}{R_0^{n+2}} R^n + \dfrac{nR^n R_0^{n-1}}{R_0^{2n+1}} \right] P_n \left( x \right) P_n \left( x' \right) \\
    &= - \dfrac{1}{4\pi \varepsilon_0} \sum_n \left[ \dfrac{n+1}{R_0^{n+2}} R^n + \dfrac{nR^n }{R_0^{n+2}} \right] P_n \left( x \right) P_n \left( x' \right) \\
    &= - \dfrac{1}{4\pi \varepsilon_0} \sum_n \left[ \left( 2n+1 \right) \dfrac{R^n }{R_0^{n+2}} \right] P_n \left( x \right) P_n \left( x' \right)
  \end{aligned}
\end{equation*}

空间中电势的解为

\begin{equation*}
  \begin{aligned}
    \varphi \left( x \right) = - \varepsilon_0 \oint_S \varphi \left( x' \right) \dfrac{\partial G}{\partial R'}  \md S'
  \end{aligned}
\end{equation*}

其中

\begin{equation*}
  \begin{aligned}
    \md S' = R_0^2 \sin \theta' \md \theta' \md \phi' = R_0^2 \md x' \md \phi'
  \end{aligned}
\end{equation*}

所以

\begin{equation*}
  \begin{aligned}
    \varphi \left( x \right) &= - \varepsilon_0 \oint_S \varphi \left( x' \right) \dfrac{\partial G}{\partial R'} R_0^2 \md x' \md \phi'
    = - 2 \pi \varepsilon_0 R_0^2 \int_{-1}^{+1} \varphi \left( x' \right) \dfrac{\partial G}{\partial R'} \md x' \\
    &= - 2 \pi \varepsilon_0 R_0^2 \int_{-1}^{+1} \varphi \left( x' \right) \left[ - \dfrac{1}{4\pi \varepsilon_0} \sum_n \left[ \left( 2n+1 \right) \dfrac{R^n }{R_0^{n+2}} \right] P_n \left( x \right) P_n \left( x' \right) \right] \md x' \\
    &= - 4 \pi \varepsilon_0 R_0^2 \int_0^{+1} \varphi_0  \left[ - \dfrac{1}{4\pi \varepsilon_0} \sum_n \left[ \left( 2n+1 \right) \dfrac{R^n }{R_0^{n+2}} \right] P_n \left( x \right) P_n \left( x' \right) \right] \md x' \\
    &= \int_0^1 \varphi_0  \left[ \sum_n \left[ \left( 2n+1 \right) \dfrac{R^n }{R_0^n} \right] P_n \left( x \right) P_n \left( x' \right) \right] \md x' \\
    &= \varphi_0 \sum_n  \left[ \left( 2n+1 \right) \dfrac{R^n }{R_0^n} \right] P_n \left( x \right) \int_0^1  P_n \left( x' \right) \md x' \\
    &= \varphi_0 \sum_n  \left[  \dfrac{R^n }{R_0^n} \right] P_n \left( x \right) \left( 2n+1 \right) \int_0^1  P_n \left( x' \right) \md x' \\
    &= \varphi_0 \sum_n  \left[ \dfrac{R^n }{R_0^n} \right] P_n \left( x \right) \left[ \left( -1  \right)^{\left( n-1 \right)/ 2} \cdot \dfrac{1 \cdot 3 \cdot 5 \dots \left( n-2 \right)}{2 \cdot 4 \cdot 6 \dots \left( n-1 \right)} \cdot \dfrac{2n+1}{n+1} \right] \\
    &= \varphi_0 \sum_n  \left[ \left( -1  \right)^{\left( n-1 \right)/ 2} \cdot \dfrac{1 \cdot 3 \cdot 5 \dots \left( n-2 \right)}{2 \cdot 4 \cdot 6 \dots \left( n-1 \right)} \cdot \dfrac{2n+1}{n+1} \right] \dfrac{R^n }{R_0^n} P_n \left( x \right) 
  \end{aligned}
\end{equation*}

其中$n$为奇数。在球内,电势为

\begin{equation*}
  \begin{aligned}
    \varphi_i &=
    \varphi_0 \sum_{n=0}^{n=+\infty}  \left[ \left( -1  \right)^{\left( n-1 \right)/ 2} \cdot \dfrac{1 \cdot 3 \cdot 5 \dots \left( n-2 \right)}{2 \cdot 4 \cdot 6 \dots \left( n-1 \right)} \cdot \dfrac{2n+1}{n+1} \right] \dfrac{R^n }{R_0^n} P_n \left( x \right) && R<R_0,n=2k+1
  \end{aligned}
\end{equation*}

在球外,电势为

\begin{equation*}
  \begin{aligned}
    \varphi_o &=
    \varphi_0 \sum_{n=- \infty}^{n=-1}  \left[ \left( -1  \right)^{\left( n-1 \right)/ 2} \cdot \dfrac{1 \cdot 3 \cdot 5 \dots \left( n-2 \right)}{2 \cdot 4 \cdot 6 \dots \left( n-1 \right)} \cdot \dfrac{2n+1}{n+1} \right] \dfrac{R^n }{R_0^n} P_n \left( x \right) \\
    &= \varphi_0 \sum_{n=+1}^{n=+ \infty}  \left[ \left( -1  \right)^{\left( n-1 \right)/ 2} \cdot \dfrac{1 \cdot 3 \cdot 5 \dots \left( n-2 \right)}{2 \cdot 4 \cdot 6 \dots \left( n-1 \right)} \cdot \dfrac{2n+1}{n+1} \right] \dfrac{R_0^n }{R^n} P_n \left( x \right) \\
    &= \varphi_0 \sum_{n=0}^{n=+ \infty}  \left[ \left( -1  \right)^{\left( n-1 \right)/ 2} \cdot \dfrac{1 \cdot 3 \cdot 5 \dots \left( n-2 \right)}{2 \cdot 4 \cdot 6 \dots \left( n-1 \right)} \cdot \dfrac{2n+1}{n+1} \right] \dfrac{R_0^{n+1} }{R^{n+1}} P_n \left( x \right) && R>R_0,n=2k+1
  \end{aligned}
\end{equation*}

\end{document} 
