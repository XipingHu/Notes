\documentclass{article}

% Chinese Support using xeCJK
% \usepackage{xeCJK}
% \setCJKmainfont{SimSun}

% Chinese Support using CTeX
\usepackage{ctex}

% Math Support
\usepackage{amsmath}
\usepackage{amsfonts}
\usepackage{amssymb}
\usepackage{wasysym}
\newcommand{\angstrom}{\text{\normalfont\AA}}

\usepackage{fancyhdr}

% Graphics Support
\usepackage{graphicx}
\usepackage{float}

% Reduced page margin
\usepackage{geometry}
\geometry{a4paper,scale=0.8}

\usepackage{caption}
\usepackage{subcaption}

% d and e should be math operators
\newcommand*{\dif}{\mathop{}\!\mathrm{d}}
\newcommand*{\md}{\mathop{}\!\mathrm{d}}
\newcommand*{\me}{\mathrm{e}}
\newcommand*{\mh}{\mathrm{h}}
\newcommand*{\Jmath}{J}

% No indent for each paragraph
\usepackage{parskip}
\setlength{\parindent}{0cm}

% Bold style for Greek letters
\usepackage{bm}
\let\Oldmathbf\mathbf
\renewcommand{\mathbf}[1]{\boldsymbol{\Oldmathbf{#1}}}

% More space for dfrac in cell
\usepackage{cellspace}
\setlength{\cellspacetoplimit}{5pt}
\setlength{\cellspacebottomlimit}{5pt}

% SI units
\newcommand{\si}[1]{\  \mathrm{#1}}

% Multi-line author information
\usepackage{authblk}
\author{物理(4+4)1801 \quad 胡喜平 \quad 学号 U201811966}

\affil{网站 https://hxp.plus/ \quad 邮件 hxp201406@gmail.com}

\title{《电动力学》课后习题——第一章\ 电磁现象的基本规律}

\pagestyle{fancy}
\fancyhf{}
\lhead{源码地址:https://github.com/hxp-plus/Notes/blob/master/Electro-Dynamics/Homework/}
\rfoot{第 \thepage 页}
\renewcommand{\headrulewidth}{1pt}
\renewcommand{\footrulewidth}{1pt}

\begin{document}

\maketitle\thispagestyle{fancy}

\paragraph{2.1}

半径为$R$的电介质球,极化强度为$\vec{P}=K \dfrac{\vec{r}}{r^2} $,电容率为$\epsilon$

(1)计算束缚电荷的体密度和面密度

(2)计算自由电荷体密度

(3)计算球外和球内的电势

(4)求该带电介质球产生的静电场总能量

\paragraph{解}



\paragraph{2.2}

贼均匀外电场中置入半径为$R_0$的导体球,试用分离变量法求下列两种情况的电势:

(1)导体球上接有电池,使球与地保持电势差$\varphi$

(2)导体球上带总电荷$Q$

\paragraph{解}



\paragraph{2.4}

均匀介质球(电容率为$\varepsilon_1$)的中心置一自由电偶极子$\vec{p}_f$,球外充满了另一种介质(电容率为$\varepsilon_2$),求空间各点电势和极化电荷分布。

提示:$\varphi = \dfrac{\vec{p}_f \cdot \vec{R}}{ 4 \pi \varepsilon_1 R^3} + \varphi' $,而$\varphi'$满足拉普拉斯方程

\paragraph{解}



\paragraph{2.8}

半径为$R_0$的导体球外充满均匀绝缘介质$\varepsilon$,导体球接地,离球心$a$处($a>R_0$)置一点电荷$Q_f$,试用分离变量法求空间各点电势,证明所得结果与镜像法结果相同

\paragraph{解}



\paragraph{2.11}

在接地的导体平面上有一半径为$a$的半球凸部,半球的球心在导体平面上,点电荷$Q$位于系统的对称轴上,并与平面相距为$b$($b>a$),试用镜像法求空间电势

\paragraph{解}




\paragraph{2.12}

有一点电荷$Q$位于两个相互垂直的接地导体平面所围成的直角空间内,它到两个平面的距离为$a$和$b$,求空间电势

\paragraph{解}




\paragraph{2.18}

一半径为$R_0$的球面,在球坐标$0<\theta< \dfrac{\pi}{2} $的半球面上电势为$\varphi_0$,在$\dfrac{\pi}{2} < \theta < \pi $的半球面上电势为$-\varphi_0$,球空间各点电势

\paragraph{解}




\paragraph{2.19}

上题能用格林函数解吗?结果如何?

\paragraph{解}


\end{document} 
