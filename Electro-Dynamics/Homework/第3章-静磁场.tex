\documentclass{article}

% Chinese Support using xeCJK
% \usepackage{xeCJK}
% \setCJKmainfont{SimSun}

% Chinese Support using CTeX
\usepackage{ctex}

% Math Support
\usepackage{amsmath}
\usepackage{amsfonts}
\usepackage{amssymb}
\usepackage{wasysym}
\newcommand{\angstrom}{\text{\normalfont\AA}}

\usepackage{fancyhdr}

% Graphics Support
\usepackage{graphicx}
\usepackage{float}

% Reduced page margin
\usepackage{geometry}
\geometry{a4paper,scale=0.8}

\usepackage{caption}
\usepackage{subcaption}

% d and e should be math operators
\newcommand*{\dif}{\mathop{}\!\mathrm{d}}
\newcommand*{\md}{\mathop{}\!\mathrm{d}}
\newcommand*{\me}{\mathrm{e}}
\newcommand*{\mh}{\mathrm{h}}
\newcommand*{\Jmath}{J}

% No indent for each paragraph
\usepackage{parskip}
\setlength{\parindent}{0cm}

% Bold style for Greek letters
\usepackage{bm}
\let\Oldmathbf\mathbf
\renewcommand{\mathbf}[1]{\boldsymbol{\Oldmathbf{#1}}}

% More space for dfrac in cell
\usepackage{cellspace}
\setlength{\cellspacetoplimit}{5pt}
\setlength{\cellspacebottomlimit}{5pt}

% SI units
\newcommand{\si}[1]{\  \mathrm{#1}}

% Multi-line author information
\usepackage{authblk}
\author{物理(4+4)1801 \quad 胡喜平 \quad 学号 U201811966}

\affil{网站 https://hxp.plus/ \quad 邮件 hxp201406@gmail.com}

\title{《电动力学》课后习题——第三章\ 静磁场}

\pagestyle{fancy}
\fancyhf{}
\lhead{源码地址:https://github.com/hxp-plus/Notes/blob/master/Electro-Dynamics/Homework/}
\rfoot{第 \thepage 页}
\renewcommand{\headrulewidth}{1pt}
\renewcommand{\footrulewidth}{1pt}

\begin{document}

\maketitle\thispagestyle{fancy}

\paragraph{3.1}

试用$\vec{A}$表示一个沿着$z$方向的均匀的恒定磁场$\vec{B}_0$,写出$\vec{A}$的两种不同的表示式,证明二者之差是无旋场。

\paragraph{解}

磁矢势的定义为

\begin{equation*}
  \nabla \times \vec{A} = \vec{B}
\end{equation*}

其中

\begin{equation*}
  \vec{B} = B_0 \hat{z}
\end{equation*}

所以

\begin{equation*}
  \nabla \cdot \vec{A} =
  \left( \frac{\partial A_{z}}{\partial y} - \frac{\partial A_{y}}{\partial z} \right) \vec{\imath}
  + \left( \frac{\partial A_{x}}{\partial z} - \frac{\partial A_{z}}{\partial x} \right) \vec{\jmath}
  + \left( \frac{\partial A_{y}}{\partial x} - \frac{\partial A_{x}}{\partial y} \right) \vec{k}
\end{equation*}

其中两组磁矢势为

\begin{equation*}
  \begin{aligned}
    \vec{A}_1 &= -B_0 y \hat{x} + 2 z \hat{y} + 2 y \hat{z} \\
    \vec{A}_2 &= 2 z \hat{x} + B_0 x \hat{y} + 2 x \hat{z}
  \end{aligned}
\end{equation*}

两组磁矢势的差为

\begin{equation*}
  \begin{aligned}
    \vec{A}_1 - \vec{A}_2 = 
    \left( - B_{0} y - 2z \right) \hat{x} + \left( 2z - B_0 x \right) \hat{y} + \left( 2y - 2x \right) \hat{z}
  \end{aligned}
\end{equation*}

很容易看出

\begin{equation*}
  \begin{aligned}
    \nabla \times \left( \vec{A}_1 - \vec{A}_2 \right) = 0
  \end{aligned}
\end{equation*}

\paragraph{3.3}

假设有无穷长的线电流$I$沿着$z$轴流动,以$z<0$空间充满磁导率为$\mu$的均匀介质,$z>0$区域为真空,试用唯一性定理求次感应强度$\vec{B}$,然后求解磁化电流分布。

\paragraph{解}

使用柱坐标,介质在下真空在上,交界处$z=0$,空间中的点和原点距离为$r$。磁场环路定理为

\begin{equation*}
  \begin{aligned}
    \oint \vec{H} \cdot \md \vec{l} = \mu_0 I
  \end{aligned}
\end{equation*}

空间磁场强度的分布为

\begin{equation*}
  \begin{aligned}
    \vec{H} = \frac{I}{2 \pi r}
  \end{aligned}
\end{equation*}

因此磁感应强度为

\begin{equation*}
  \begin{aligned}
    B = 
  \end{aligned}
  \left\{
  \begin{aligned}
    & \frac{\mu_{0} I}{2\pi r} && z>0 \\
    & \frac{\mu I}{2 \pi r} && z<0
  \end{aligned}
  \right.
\end{equation*}

磁感应强度的方向由右手螺旋定则确定。取一个四角坐标为$(r,\delta z)$、$(r+\delta r, \delta z)$、$(r+\delta r, -\delta z)$、$(r, -\delta z)$的小积分面,当$z>0$时有磁极化电流,密度为$\alpha_M$,则

\begin{equation*}
  \begin{aligned}
    \left( \frac{\mu_{0} I}{2\pi r} - \frac{\mu I}{2\pi r} \right) \times 2 \delta z \delta r
    = \mu_0 \alpha_M \times 2 \delta z \delta r
  \end{aligned}
\end{equation*}

因此

\begin{equation*}
  \begin{aligned}
    \alpha_M = \frac{I}{2\pi r} \left( 1 - \frac{\mu}{\mu_{0}} \right)
  \end{aligned}
\end{equation*}

方向为$\vec{e}_r$的反方向,因此

\begin{equation*}
  \begin{aligned}
    \vec{\alpha}_M = \frac{I}{2\pi r} \left( \frac{\mu}{\mu_{0}} - 1 \right) \vec{e}_r
  \end{aligned}
\end{equation*}

\paragraph{3.7}

半径为$a$的无限长圆柱导体上有恒定电流$\vec{\Jmath}$均匀分布于截面上,试解磁矢势$\vec{A}$的微分方程,设导体的磁导率为$\mu_0$,导体外的磁导率为$\mu$

\paragraph{解}

体系是上下平移对称的,只需要极坐标。极坐标下的拉普拉斯算子为

\begin{equation*}
  \begin{aligned}
    \nabla^2 = \frac{1}{r} \frac{\partial}{\partial r} \left( r \frac{\partial}{\partial r} \right)
    + \frac{1}{r} \frac{\partial}{\partial \theta} \left( \frac{1}{r} \frac{\partial}{\partial \theta} \right)
    = \frac{\partial^{2}}{\partial r^2} + \frac{1}{r} \frac{\partial}{\partial r} + \frac{1}{r^{2}} \frac{\partial^{2}}{\partial \theta^2}
  \end{aligned}
\end{equation*}

磁矢势微分方程为

\begin{equation*}
  \begin{aligned}
    \nabla^2 \vec{A} = - \mu \vec{\Jmath}
  \end{aligned}
\end{equation*}

由于系统轴对称

\begin{equation*}
  \begin{aligned}
    & \frac{1}{r} \frac{\partial A_i}{\partial r} \left( r \frac{\partial A_{i}}{\partial r} \right) = - \mu_0 J \\
    & \frac{1}{r} \frac{\partial A_o}{\partial r} \left( r \frac{\partial A_{o}}{\partial r} \right) = 0
  \end{aligned}
\end{equation*}

$A_o$是导体外磁矢势,$A_i$是导体内磁矢势,当$r=a$时,边值关系为

\begin{equation*}
  \begin{aligned}
    & \vec{A}_i = \vec{A}_o \\
    & \hat{r} \times \left( \frac{1}{\mu} \nabla \times \vec{A}_o \right) - \hat{r} \times \left( \frac{1}{\mu_{0}} \nabla \times A_i \right) = \alpha_M
  \end{aligned}
\end{equation*}

由于系统是轴对称的,而且电流只有径向方向,且磁矢势的零点选择在任何地方都不影响,在边界

\begin{equation*}
  \begin{aligned}
    & A_i = A_o = 0 \\
    & \frac{1}{\mu} \frac{\md A_{o}}{\md r} = \frac{1}{\mu_{0}} \frac{\md A_{i}}{\md r}
  \end{aligned}
\end{equation*}

对磁矢势微分方程进行积分,得到

\begin{equation*}
  \begin{aligned}
    & r \frac{\partial A_{i}}{\partial r} + \frac{1}{2} \mu_0 J r^2 = C_1
    \Rightarrow \frac{\partial A_{i}}{\partial r} = - \frac{1}{2} \mu_0 J r - \frac{C_{1}}{r}
    \Rightarrow A_i = - \frac{\mu_0}{4} J r^2 - C_1 \ln r + C_3
    \\
    & r \frac{\partial A_{o}}{\partial r} = C_2
    \Rightarrow \frac{\partial A_{o}}{\partial r} = \frac{C_{2}}{r}
    \Rightarrow A_o = C_2 \ln r + C_4
  \end{aligned}
\end{equation*}

带入$r=a$时的边界条件和$r=0$时磁矢势是有限值的条件

\begin{equation*}
  \begin{aligned}
    & C_1 = 0 \\
    & - \frac{\mu_0}{4} J a^2 - C_1 \ln a + C_3 = 0 \\
    & C_2 \ln a + C_4 = 0 \\
    & \frac{C_2 / a}{\mu} = \frac{- \frac{\mu_0}{2} J a - C_1 / a}{\mu_{0}}
  \end{aligned}
\end{equation*}

解得

\begin{equation*}
  \begin{aligned}
    & C_1 = 0 \\
    & C_2 = - \frac{\mu}{2} J a^2 \\
    & C_3 = \frac{\mu_{0}}{4} J a^2 \\
    & C_4 = \frac{\mu}{2} J a^2 \ln a
  \end{aligned}
\end{equation*}

所以

\begin{equation*}
  \begin{aligned}
    A_i &= \frac{\mu_{0}}{4} J \left( a^2 - r^2 \right) \\
    A_o &= \frac{\mu}{2} J a^2 \left( \ln a - \ln r \right)
  \end{aligned}
\end{equation*}

\paragraph{3.9}

将一个磁导率为$\mu$,半径为$R_0$的球体,放入均匀磁场$\vec{H}_0$内,求总磁感应强度$\vec{B}$和诱导磁矩$\vec{m}$

\paragraph{解}

用磁标势,球里面的磁标势为$\varphi_i$,球外边的磁标势为$\varphi_o$

\begin{equation*}
  \begin{aligned}
    \nabla^2 \varphi_i = \nabla^2 \varphi_o = 0
  \end{aligned}
\end{equation*}

$r=R_0$时有边界条件

\begin{equation*}
  \begin{aligned}
    \varphi_i &= \varphi_o \\
    \mu \frac{\partial \varphi_{i}}{\partial r} &= \mu_0 \frac{\partial \varphi_{o}}{\partial r}
  \end{aligned}
\end{equation*}

设

\begin{equation*}
  \begin{aligned}
    \varphi_i &= \sum_n a_n r^n P_n \left( \cos \theta \right) \\
    \varphi_o &= - H_0 r P_1 \left( \cos \theta \right) + \sum_n \frac{b_{n}}{r^{n+1}} P_n \left( \cos \theta \right)
  \end{aligned}
\end{equation*}

代入边界条件

\begin{equation*}
  \begin{aligned}
    \sum_n a_n R^n P_n \left( \cos \theta \right) &= - H_0 R P_1 \left( \cos \theta \right) + \sum_n \frac{b_{n}}{R^{n+1}} P_n \left( \cos \theta \right) \\
    \frac{\mu}{\mu_{0}} \sum_n n a_n R^{n-1} P_n \left( \cos \theta \right) &= - H_0 P_1 \left( \cos \theta \right) - \sum_n \left( n + 1 \right) \frac{b_{n}}{R^{n+2}} P_n \left( \cos \theta \right)
  \end{aligned}
\end{equation*}

比较系数得出

\begin{equation*}
  \begin{aligned}
    a_1 R &= - H_0 R + \frac{b_{1}}{R^{2}} \\
    \frac{\mu}{\mu_{0}} a_1 &= - H_0 - 2 \frac{b_{1}}{R^{3}}
  \end{aligned}
\end{equation*}

和

\begin{equation*}
  \begin{aligned}
    a_n = b_n = 0 && n \neq 1
  \end{aligned}
\end{equation*}

解得

\begin{equation*}
  \begin{aligned}
    a_1 &= - \frac{3 \mu_{0}}{2 \mu_{0} + \mu} H_0 \\
    b_1 &= \frac{\mu - \mu_{0}}{2\mu_0 + \mu} H_0 R_0^3 \\
    a_n &= b_n = 0 && n \neq 1
  \end{aligned}
\end{equation*}

带回磁标势表达式

\begin{equation*}
  \begin{aligned}
    \varphi_i &= \sum_n a_n r^n P_n \left( \cos \theta \right)
    = - \frac{3 \mu_{0}}{2 \mu_{0} + \mu} H_0 r \cos \theta \\
    \varphi_o &= - H_0 r P_1 \left( \cos \theta \right) + \sum_n \frac{b_{n}}{r^{n+1}} P_n \left( \cos \theta \right)
    = - H_0 r \cos \theta + \frac{\mu - \mu_{0}}{2\mu_0 + \mu} \cdot \frac{H_0 R_0^3 \cos \theta}{r^{2}}
  \end{aligned}
\end{equation*}

重写表达式为矢量形式

\begin{equation*}
  \begin{aligned}
    \varphi_i &= - \frac{3 \mu_{0}}{2 \mu_{0} + \mu} \vec{H}_0 \cdot \vec{r} \\
    \varphi_o &= - \vec{H}_0 \cdot \vec{r} + \frac{\mu - \mu_{0}}{2\mu_0 + \mu} \cdot \frac{R_0^3 }{r^{3}} \vec{H}_0 \cdot \vec{r}
  \end{aligned}
\end{equation*}

求梯度得到磁感应强度

\begin{equation*}
  \begin{aligned}
    \vec{B}_i &= - \mu \nabla \varphi_i = \frac{3 \mu \mu_{0}}{2\mu_{0} + \mu} \vec{H}_0 \\
    \vec{B}_o &= - \mu_0 \nabla \varphi_o = \mu_0 \vec{H}_0 - \frac{\mu \left( \mu - \mu_0 \right) R_0^3}{2 \mu_0  \mu} \left[ \frac{\vec{H}}{r^{3}} - \frac{3 \left( \vec{H} \cdot \vec{r} \right) \vec{r}}{r^5} \right] 
  \end{aligned}
\end{equation*}

介质的磁极化强度

\begin{equation*}
  \begin{aligned}
    \vec{M} = \left( \frac{\mu}{\mu_{0}} - 1 \right) \frac{B_{i}}{\mu}
    = \left(\frac{\mu}{\mu_{0}} - 1 \right) \frac{3 \mu_{0}}{2\mu_{0} + \mu} \vec{H}_0 
    = \frac{3 \left( \mu - \mu_0 \right)}{2\mu_{0} + \mu} \vec{H}_0 
  \end{aligned}
\end{equation*}

磁化电流为

\begin{equation*}
  \begin{aligned}
    \frac{4}{3} \pi R_0^3 \vec{M} = \frac{4 \pi R_0^3 \left( \mu - \mu_0 \right)}{2\mu_{0} + \mu} \vec{H}_0
  \end{aligned}
\end{equation*}

\paragraph{3.13}

有一个均匀带电的薄导体壳,半径为$R_0$,总电荷为$Q$,使球壳自身绕着某一直径以角速度$\omega$转动,求球内外磁场$\vec{B}$

\paragraph{解}

在球外,可以将球看成一个个环形电流,球面上线电流密度为

\begin{equation*}
  \begin{aligned}
    \alpha_M = \frac{q \omega}{4 \pi R_0^2} R_0 \sin \theta \vec{e}_{\phi}
  \end{aligned}
\end{equation*}

总的磁偶极矩是

\begin{equation*}
  \begin{aligned}
    m = \int_0^{\pi} \pi \left( R_0 \sin \theta \right)^2 \alpha_M R_0 \md \theta
    = \frac{q \omega}{4} R_0^2 \int_0^{\pi} \sin^3 \theta \md \theta
    = \frac{q \omega R_0^2}{3}
  \end{aligned}
\end{equation*}

方向是和转轴的方向相同。因此球外的磁标势为

\begin{equation*}
  \begin{aligned}
    \varphi_o = \frac{\vec{m} \cdot \vec{R}}{4\pi R^{3}} = \frac{m \cos \theta}{4 \pi R^{2}}
  \end{aligned}
\end{equation*}

磁感应强度为

\begin{equation*}
  \begin{aligned}
    \vec{B}_o = - \mu_0 \nabla \varphi_o = \frac{\mu_{0}}{4\pi} \left[ \frac{3 \left( \vec{m} \cdot \vec{R} \right)\vec{R}}{R^{5}} - \frac{\vec{m}}{R^{3}} \right] 
  \end{aligned}
\end{equation*}

设球内磁标势为

\begin{equation*}
  \begin{aligned}
    \varphi_i = \sum_n a_{n} R^{n} P_n \left( \cos \theta \right)
  \end{aligned}
\end{equation*}

因为球内和球外交界处有自由电流,不能引入磁标势,但是可以用g磁场边值关系。法向磁场的连续性条件为

\begin{equation*}
  \begin{aligned}
    & \sum_n n a_{n} R_0^{n-1} P_n \left( \cos \theta \right) = - 2 \frac{m \cos \theta}{4\pi R_0^{3}}
  \end{aligned}
\end{equation*}

因此

\begin{equation*}
  \begin{aligned}
    a_1 &= - \frac{m}{2\pi R_0^3} \\
    a_n &= 0 && n \neq 1
  \end{aligned}
\end{equation*}

因此球内磁标势为

\begin{equation*}
  \begin{aligned}
    \varphi_i = - \frac{m}{2\pi R_0^3} R \cos \theta = - \frac{\vec{m} \cdot \vec{R}}{2\pi R_{0}^{3}}
  \end{aligned}
\end{equation*}

磁感应强度为

\begin{equation*}
  \begin{aligned}
    \vec{B}_i = - \mu_0 \nabla \varphi_i = \frac{\mu_0 \vec{m}}{2\pi R_{0}^{3}}
  \end{aligned}
\end{equation*}

\end{document} 




%%% Local Variables:
%%% mode: latex
%%% TeX-master: t
%%% End:
