\documentclass{article}

% Chinese Support using xeCJK
% \usepackage{xeCJK}
% \setCJKmainfont{SimSun}

% Chinese Support using CTeX
\usepackage{ctex}

% Math Support
\usepackage{amsmath}
\usepackage{amsfonts}
\usepackage{amssymb}
\usepackage{wasysym}
\newcommand{\angstrom}{\text{\normalfont\AA}}

\usepackage{fancyhdr}

% Graphics Support
\usepackage{graphicx}
\usepackage{float}

% Reduced page margin
\usepackage{geometry}
\geometry{a4paper,scale=0.8}

\usepackage{caption}
\usepackage{subcaption}

% d and e should be math operators
\newcommand*{\dif}{\mathop{}\!\mathrm{d}}
\newcommand*{\md}{\mathop{}\!\mathrm{d}}
\newcommand*{\me}{\mathrm{e}}
\newcommand*{\mh}{\mathrm{h}}
\newcommand*{\Jmath}{J}

% No indent for each paragraph
\usepackage{parskip}
\setlength{\parindent}{0cm}

% Bold style for Greek letters
\usepackage{bm}
\let\Oldmathbf\mathbf
\renewcommand{\mathbf}[1]{\boldsymbol{\Oldmathbf{#1}}}

% More space for dfrac in cell
\usepackage{cellspace}
\setlength{\cellspacetoplimit}{5pt}
\setlength{\cellspacebottomlimit}{5pt}

% SI units
\newcommand{\si}[1]{\  \mathrm{#1}}

% Multi-line author information
\usepackage{authblk}
\author{物理(4+4)1801 \quad 胡喜平 \quad 学号 U201811966}

\affil{网站 https://hxp.plus/ \quad 邮件 hxp201406@gmail.com}

\title{《电动力学》课后习题——第一章\ 电磁现象的基本规律}

\pagestyle{fancy}
\fancyhf{}
\lhead{源码地址:https://github.com/hxp-plus/Notes/blob/master/Eletro-Dynamics/Homework/}
\rfoot{第 \thepage 页}
\renewcommand{\headrulewidth}{1pt}
\renewcommand{\footrulewidth}{1pt}

\begin{document}

\maketitle\thispagestyle{fancy}

\paragraph{1.1}

根据算符$\nabla$的微分性与矢量性,推导下列公式:

\begin{equation}
  \begin{aligned}
    \label{eq:q1}
    \nabla \left( \vec{A} \cdot \vec{B} \right) = \vec{B} \times \left( \nabla \cdot \vec{A} \right) + \left( \vec{B} \cdot \nabla \right) \vec{A} + \vec{A} \times \left( \nabla \times \vec{B}  \right) 
  \end{aligned}
\end{equation}

\begin{equation}
  \label{eq:q2}
  \begin{aligned}
    \vec{A} \times \left( \nabla \times \vec{A} \right) = \dfrac{1}{2} \nabla A^2 - \left( \vec{A} \cdot \nabla \right) \vec{A} 
  \end{aligned}
\end{equation}

\paragraph{解}

\begin{equation}
  \label{eq:1}
  \begin{aligned}
    \nabla \left( \vec{A} \cdot \vec{B} \right) = \left( \partial_i \vec{e}_i \right) \left( A_j B_j  \right) = \left( A_j \partial_i B_j + B_j \partial_i A_j \right) \vec{e}_i
  \end{aligned}
\end{equation}

\begin{equation}
  \label{eq:2}
  \begin{aligned}
    \left( \vec{B} \cdot \nabla \right) \vec{A} = \left( B_i \vec{e}_i \cdot \partial_j \vec{e}_j  \right) \vec{A} = \left( \delta_{ij} B_i \partial_j \right) \vec{A} = \left( B_i \partial_i  \right) \left( A_j \vec{e}_j \right) = B_i \partial_i A_j \vec{e}_j
  \end{aligned}
\end{equation}

同理

\begin{equation}
  \label{eq:3}
  \begin{aligned}
    \left( \vec{A} \cdot \nabla \right) \vec{B} = A_i \partial_i B_j \vec{e}_j
  \end{aligned}
\end{equation}

\begin{equation}
  \label{eq:4}
  \begin{aligned}
    \vec{B} \times \left( \nabla \times \vec{A} \right) &= \vec{B} \times \left( \epsilon_{ijk} \partial_i A_j \vec{e}_k \right) = \epsilon_{mnl} B_m \left( \epsilon_{ijk} \partial_i A_j \vec{e}_k \right)_n \vec{e}_l = \epsilon_{mnl} B_m \epsilon_{ijn} \partial_i A_j \vec{e}_l = \epsilon_{lmn} \epsilon_{ijn} B_m \partial_i A_j \vec{e}_l\\ &= \left( B_m \partial_l A_m - B_m \partial_m A_l \right) \vec{e}_l
  \end{aligned}
\end{equation}

同理

\begin{equation}
  \label{eq:5}
  \begin{aligned}
    \vec{A} \times \left( \nabla \times \vec{B} \right) = \left( A_m \partial_l B_m - A_m \partial_m B_l \right) \vec{e}_l
  \end{aligned}
\end{equation}

式(\ref{eq:2})(\ref{eq:3})(\ref{eq:4})(\ref{eq:5})相加,显然等于式(\ref{eq:1}),因此式(\ref{eq:q1})得证。

\begin{equation}
  \label{eq:6}
  \begin{aligned}
    \vec{A} \times \left( \nabla \times \vec{A} \right) &= \vec{A} \times \left[ \left( \partial_i \vec{e}_i \right) \times \left( A_j \vec{e}_j \right) \right] = \vec{A} \times \left( \epsilon_{ijk} \partial_i A_j \vec{e}_k \right) = \left( A_l \vec{e}_l \right) \times \left( \epsilon_{ijk} \partial_{ijk} \partial_i A_j \vec{e}_k \right) \\ &= \epsilon_{ijk} \epsilon_{lkn} A_l \partial_i A_j \vec{e}_n = \epsilon_{ijk} \epsilon_{nlk} A_l \partial_i A_j \vec{e}_n = \left( \delta_{in} \delta_{jl} - \delta_{il} \delta_{jn} \right) A_l \partial_i A_j \vec{e}_n \\ &= A_j \partial_i A_j \vec{e}_i 
  \end{aligned}
\end{equation}

\begin{equation}
  \label{eq:7}
  \begin{aligned}
    \left( \vec{A} \cdot \nabla \right) \vec{A} = \left( A_i \partial_i \right) \left( A_j \vec{e}_j \right) = A_j \partial_i A_j \vec{e}_j
  \end{aligned}
\end{equation}

显然,式(\ref{eq:6})和(\ref{eq:7})是相等的,得证。

\paragraph{1.2}

设$u$是空间坐标$x,y,z$的函数,证明:

\begin{equation*}
  \begin{aligned}
    \nabla f \left( u \right) = \dfrac{\md f}{\md u} \nabla u 
  \end{aligned}
\end{equation*}

\begin{equation*}
  \begin{aligned}
    \nabla \cdot \vec{A} \left( u \right) = \nabla u \cdot \dfrac{\md \vec{A}}{\md u} 
  \end{aligned}
\end{equation*}

\begin{equation*}
  \begin{aligned}
    \nabla \times \vec{A} \left( u \right) = \nabla u \times \dfrac{\md \vec{A}}{\md u} 
  \end{aligned}
\end{equation*}

\paragraph{解}

\begin{equation*}
  \begin{aligned}
    \nabla f = \partial_i f_i \vec{e}_i = \dfrac{\partial}{\partial x_i} f_i \vec{e}_i = \dfrac{\partial f_i}{\partial u} \dfrac{\partial u}{\partial x_i} \vec{e}_i = \dfrac{\md f}{\md u} \nabla u     
  \end{aligned}
\end{equation*}

\begin{equation*}
  \begin{aligned}
    \nabla \cdot \vec{A} = \partial_i A_i = \dfrac{\partial A_i}{\partial x_i} = \dfrac{\partial A_i}{\partial _u} \dfrac{\partial u}{\partial x_i} = \nabla u \cdot \dfrac{\md \vec{A}}{\md u}   
  \end{aligned}
\end{equation*}

\begin{equation*}
  \begin{aligned}
    \nabla \times \vec{A} = \epsilon_{ijk} \partial_i A_j \vec{e}_k = \epsilon_{ijk} \dfrac{\partial u}{\partial x_i} \dfrac{\partial A_j}{\partial u} \vec{e}_k = \nabla u \times \dfrac{\md \vec{A}}{\md u}  
  \end{aligned}
\end{equation*}

\paragraph{1.3}

设$r=\sqrt{\left( x - x' \right)^2 + \left( y - y' \right)^2 + \left( z - z' \right)^2}$为源点$x'$到$x$的距离,$\vec{r}$的方向规定为源点指向场点。

(1)证明以下结果,并体会对源变数求微商($\nabla' = \vec{e}_i \dfrac{\partial}{\partial x_i'} $)和对场变数求微商($\nabla = \vec{e}_i \dfrac{\partial }{\partial x_i} $)的关系。

\begin{equation*}
  \begin{aligned}
    \nabla \vec{r} = - \nabla' \vec{r} = \dfrac{\vec{r}}{r} 
  \end{aligned}
\end{equation*}

\begin{equation*}
  \begin{aligned}
    \nabla \dfrac{1}{r} = - \nabla' \dfrac{1}{r} = - \dfrac{\vec{r}}{r^3}   
  \end{aligned}
\end{equation*}

\begin{equation*}
  \begin{aligned}
    \nabla \times \dfrac{\vec{r}}{r^3} = 0 
  \end{aligned}
\end{equation*}

\begin{equation*}
  \begin{aligned}
    \nabla \cdot \dfrac{\vec{r}}{r^3} = - \nabla' \cdot \dfrac{\vec{r}}{r^3} = 0  
  \end{aligned}
\end{equation*}

\paragraph{解}

因为

\begin{equation*}
  \begin{aligned}
    \nabla = \partial_i = \dfrac{\partial}{\partial x_i} = \dfrac{\partial}{\partial x'_i} \dfrac{\partial x'_i}{\partial x_i} = - \dfrac{\partial}{\partial x'_i} = - \nabla   
  \end{aligned}
\end{equation*}

所以

\begin{equation*}
  \begin{aligned}
    \nabla r = - \nabla' r
  \end{aligned}
\end{equation*}

\begin{equation*}
  \begin{aligned}
    \nabla \dfrac{1}{r} = - \nabla' \dfrac{1}{r}
  \end{aligned}
\end{equation*}

\begin{equation*}
  \begin{aligned}
    \nabla \times \dfrac{\vec{r}}{r^3} = 0 
  \end{aligned}
\end{equation*}

\begin{equation*}
  \begin{aligned}
    \nabla \cdot \dfrac{\vec{r}}{r^3} = - \nabla' \cdot \dfrac{\vec{r}}{r^3} 
  \end{aligned}
\end{equation*}

只需要证明

\begin{equation}
  \label{eq:a311}
  \begin{aligned}
    \nabla \vec{r} = \dfrac{\vec{r}}{r} 
  \end{aligned}
\end{equation}

\begin{equation}
  \label{eq:a312}
  \begin{aligned}
    \nabla \dfrac{1}{r} = - \dfrac{\vec{r}}{r^3}   
  \end{aligned}
\end{equation}

\begin{equation}
  \label{eq:a313}
  \begin{aligned}
    \nabla \times \dfrac{\vec{r}}{r^3} = 0 
  \end{aligned}
\end{equation}

\begin{equation}
  \label{eq:a314}
  \begin{aligned}
    \nabla \cdot \dfrac{\vec{r}}{r^3}  = 0  
  \end{aligned}
\end{equation}

对于(\ref{eq:a311})

\begin{equation*}
  \begin{aligned}
    \nabla r = \partial_i \sqrt{ \Sigma \left( x_i - x_i' \right)^2} \vec{e}_i = \dfrac{x_i}{r} \vec{e}_i = \dfrac{\vec{r}}{r}  
  \end{aligned}
\end{equation*}

对于(\ref{eq:a312})

\begin{equation*}
  \begin{aligned}
    \nabla \dfrac{1}{r} = - \dfrac{1}{r^2} \nabla r = - \dfrac{\vec{r}}{r^3} 
  \end{aligned}
\end{equation*}

对于(\ref{eq:a313})

\begin{equation*}
  \begin{aligned}
    \nabla \times \dfrac{\vec{r}}{r^3} = \epsilon_{ijk} \partial_i \left( \dfrac{\vec{r}}{r^3}  \right)_j \vec{e}_k = \epsilon_{ijk} \partial_i \left( \dfrac{x_j}{r^3}  \right) \vec{e}_k = 0
  \end{aligned}
\end{equation*}

对于(\ref{eq:a314})

\begin{equation*}
  \begin{aligned}
    \nabla \cdot \dfrac{\vec{r}}{r^3}  &= \partial_i \left( \dfrac{x_i}{r^3}  \right) = \dfrac{r^3 \partial_i x_i - x_i \partial_i r^3}{r^6} = \dfrac{r^3 \partial_i x_i - 3 r^2 x_i \partial_i r}{r^6 } = \dfrac{r^3 - 3 r^2 x_i \dfrac{x_i}{r} }{r^6} = \dfrac{r^3 - 3 r x_i^2  }{r^6}  \\
    &= \dfrac{r^3 - 3 r x_1^2  }{r^6} + \dfrac{r^3 - 3 r x_2^2  }{r^6} + \dfrac{r^3 - 3 r x_3^2  }{r^6} = \dfrac{3 r^3 - 3r \left( x_1^2 + x_2^2 + x_3^2 \right)}{r^6} = 0
  \end{aligned}
\end{equation*}

由于分母上有$r$,当$r=0$时,不一定成立。

(2)求$\nabla \cdot \vec{r}$,$\nabla \times \vec{r}$,$\left( \vec{a} \cdot \nabla \right) \vec{r}$,$\nabla \left( \vec{a} \cdot \vec{r} \right)$,$\nabla \cdot \left[ \vec{E}_0 \sin \left( \vec{k} \cdot \vec{r} \right) \right]$,$\nabla \times \left[ \vec{E}_0 \sin \left( \vec{k} \cdot \vec{r} \right) \right]$,$\vec{a}$、$\vec{k}$、$\vec{E}_0$是常矢量。

\paragraph{解}

\begin{equation*}
  \begin{aligned}
    \nabla \cdot \vec{r} = \partial_i r_i = 3
  \end{aligned}
\end{equation*}

\begin{equation*}
  \begin{aligned}
    \nabla \times \vec{r} = \epsilon_{ijk} \partial_i x_j \vec{e}_k = 0
  \end{aligned}
\end{equation*}

\begin{equation*}
  \begin{aligned}
    \left( \vec{a} \cdot \nabla \right) \vec{r} = a_i \partial_i x_i \vec{e}_i = a_i \vec{e}_i = \vec{a}
  \end{aligned}
\end{equation*}

\begin{equation*}
  \begin{aligned}
    \nabla \left( \vec{a} \cdot \vec{r} \right) = \partial_i \left( a_j r_j \right) \vec{e}_i = \left[ a_j \partial_i x_j + x_j \partial_i a_j \right] = a_j \partial_i x_j = a_i \partial_i x_i \vec{e}_i = a_i \vec{e}_i = \vec{a}
  \end{aligned}
\end{equation*}

\begin{equation*}
  \begin{aligned}
    \nabla \cdot \left[ \vec{E}_0 \sin \left( \vec{k} \cdot \vec{r} \right) \right] = \partial_i E_{0i} \sin \left( k_i x_i \right) = E_{0i} \cos \left( k_i x_i \right) k_i = \vec{k} \cdot \vec{E}_0 \cos \left( \vec{k} \cdot \vec{r} \right) 
  \end{aligned}
\end{equation*}

\begin{equation*}
  \begin{aligned}
    \nabla \times \left[ \vec{E}_0 \sin \left( \vec{k} \cdot \vec{r} \right) \right] = \epsilon_{ijk} \partial_i E_{0j} \sin \left( k_j x_j \right) \vec{e}_k = \epsilon_{ijk} E_{0j} \cos \left( k_m x_m \right) k_i \vec{e}_k = \vec{k} \times \vec{E}_0 \cos \left( \vec{k} \cdot \vec{r} \right)
  \end{aligned}
\end{equation*}

\paragraph{1.4}

利用高斯定理证明

\begin{equation*}
  \begin{aligned}
    \int_V \md V \nabla \times \vec{f} = \oint_S \md \vec{S} \times \vec{f} 
  \end{aligned}
\end{equation*}

利用斯托克斯定理证明

\begin{equation*}
  \begin{aligned}
    \int_S \md \vec{S} \times \nabla \varphi = \oint_L \varphi \md \vec{l}
  \end{aligned}
\end{equation*}

\paragraph{解}

引入常矢量$\vec{c}$

\begin{equation*}
  \begin{aligned}
    \int_V \md V \nabla \times \vec{f} \cdot \vec{c} = \int_V \vec{c} \cdot \left( \nabla \times \vec{f} \right) \md V = \int_V \nabla \cdot \left( \vec{f} \times \vec{c} \right) \md V = \oint_S \left( \vec{f} \times \vec{c} \right) \cdot \md \vec{S} = \oint_S \md \vec{S} \times \vec{f} \cdot \vec{c}
  \end{aligned}
\end{equation*}

\begin{equation*}
  \begin{aligned}
    \int_S \md \vec{S} \times \nabla \varphi \cdot \vec{c} = \int_S \nabla \varphi \times \vec{c} \cdot \md \vec{S} = \int_L \nabla \times \left( \varphi \vec{c} \right) \cdot \md \vec{S} = \oint_L \varphi \vec{c} \cdot \md \vec{l} = \oint_L \varphi \md \vec{l} \cdot \vec{c}
  \end{aligned}
\end{equation*}

因为$\vec{c}$是任意的

\begin{equation*}
  \begin{aligned}
    \int_V \md V \nabla \times \vec{f} = \oint_S \md \vec{S} \times \vec{f} 
  \end{aligned}
\end{equation*}

\begin{equation*}
  \begin{aligned}
    \int_S \md \vec{S} \times \nabla \varphi = \oint_L \varphi \md \vec{l}
  \end{aligned}
\end{equation*}

\paragraph{1.5}

已知一个电荷系统的电偶极矩为

\begin{equation*}
  \begin{aligned}
    \vec{p} \left( t \right) = \int_V \rho \left( \vec{x}', t \right) \vec{x}' \md V'
  \end{aligned}
\end{equation*}

利用电荷守恒定律$\nabla \cdot \vec{\Jmath} + \dfrac{\partial \rho}{\partial t} = 0 $证明

\begin{equation*}
  \begin{aligned}
    \dfrac{\md \vec{p}}{\md t} = \int_V \vec{\Jmath} \left( \vec{x}', t \right) \md V' 
  \end{aligned}
\end{equation*}

\paragraph{解}

\begin{equation*}
  \begin{aligned}
    \dfrac{\md \vec{p}}{\md t} &= \int_V \dfrac{\md \rho \left( \vec{x}', t \right) }{\md t} \vec{x}' \md V' = - \int_V \left( \nabla' \cdot \vec{\Jmath} \right) \vec{x}' \md V' = - \int_V \left( \nabla' \cdot \vec{\Jmath} \vec{x}' \right) - \vec{\Jmath} \left( \nabla' \vec{x}' \right) \md V' \\
    &= - \oint_S \vec{\Jmath} \vec{x}' \cdot \md \vec{S} + \int_V \vec{\Jmath} \md V' = \int_V \vec{\Jmath} \md V'
  \end{aligned}
\end{equation*}

\paragraph{1.6}

若$\vec{m}$是常矢量,证明除$R=0$以外,矢量$\vec{A} = \dfrac{\vec{m} \times \vec{R}}{R^3} $的旋度等于标量$\varphi = \dfrac{\vec{m} \cdot \vec{R}}{R^3} $的梯度的负值,即

\begin{equation*}
  \begin{aligned}
    \nabla \times \vec{A} = - \nabla \varphi
  \end{aligned}
\end{equation*}

\paragraph{解}

\begin{equation*}
  \begin{aligned}
    \nabla \times \vec{A} &= \nabla \times \left( \vec{m} \times \dfrac{\vec{R}}{R^3}  \right) = \left( \dfrac{\vec{R}}{R^3} \cdot \nabla  \right) \vec{m} + \left( \nabla \cdot \dfrac{\vec{R}}{R^3}  \right) \vec{m} - \left( \vec{m} \cdot \nabla \right)\dfrac{\vec{R}}{R^3} - \left( \nabla \cdot \vec{m} \right) \dfrac{\vec{R}}{R^3} \\
    &= \left( \nabla \cdot \dfrac{\vec{R}}{R^3}  \right) \vec{m} - \left( \vec{m} \cdot \nabla \right) \dfrac{\vec{R}}{R^3} = - \left( \vec{m} \cdot \nabla \right) \dfrac{\vec{R}}{R^3}  
  \end{aligned}
\end{equation*}

\begin{equation*}
  \begin{aligned}
    - \nabla \varphi &= - \nabla \left( \dfrac{\vec{m} \cdot \vec{R}}{R^3}  \right) = - \vec{m} \times \left( \nabla \times \dfrac{\vec{R}}{R^3}  \right) - \left( \vec{m} \cdot \nabla \right) \dfrac{\vec{R}}{R^3} - \dfrac{\vec{R}}{R^3} \times \left( \nabla \times \vec{m} \right) - \left( \dfrac{\vec{R}}{R^3} \cdot \nabla  \right) \vec{m} \\
    &= - \vec{m} \times \left( \nabla \times \dfrac{\vec{R}}{R^3}  \right) - \left( \vec{m} \cdot \nabla \right) \dfrac{\vec{R}}{R^3} = - \left( \vec{m} \cdot \nabla \right) \dfrac{\vec{R}}{R^3}  
  \end{aligned}
\end{equation*}

得证




\end{document} 
