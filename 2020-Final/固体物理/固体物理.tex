\documentclass{article}

% Chinese Support using xeCJK
% \usepackage{xeCJK}
% \setCJKmainfont{SimSun}

% Chinese Support using CTeX
\usepackage{ctex}

% Math Support
\usepackage{amsmath}
\usepackage{amsfonts}
\usepackage{amssymb}
\usepackage{wasysym}
\newcommand{\angstrom}{\text{\normalfont\AA}}

\usepackage{fancyhdr}

% Graphics Support
\usepackage{graphicx}
\usepackage{float}
\restylefloat{table}

% Reduced page margin
\usepackage{geometry}
\geometry{a4paper,scale=0.8}

\usepackage{caption}
\usepackage{subcaption}

% d and e should be math operators
\newcommand*{\dif}{\mathop{}\!\mathrm{d}}
\newcommand*{\md}{\mathop{}\!\mathrm{d}}
\newcommand*{\me}{\mathrm{e}}

% No indent for each paragraph
\usepackage{parskip}
\setlength{\parindent}{0cm}

% Bold style for Greek letters
\usepackage{bm}
\let\Oldmathbf\mathbf
\renewcommand{\mathbf}[1]{\boldsymbol{\Oldmathbf{#1}}}

% More space for dfrac in cell
\usepackage{cellspace}
\setlength{\cellspacetoplimit}{5pt}
\setlength{\cellspacebottomlimit}{5pt}

% SI units
\newcommand{\si}[1]{\  \mathrm{#1}}

% Multi-line author information
\usepackage{authblk}
\author{胡喜平}
\usepackage{multicol}
\title{固体物理}

\pagestyle{fancy}
\fancyhf{}
\lhead{https://hxp.plus/}
\rhead{固体物理}
\rfoot{第 \thepage 页}
\renewcommand{\headrulewidth}{1pt}
\renewcommand{\footrulewidth}{1pt}

\begin{document}


\begin{multicols}{2}
  

\section{原子的凝聚}

\subsection{原子结构}

原子的能量和波函数

\begin{equation*}
  E_n = - \dfrac{\mu Z e^{4}}{2 \hbar^{2} \left( 4 \pi \varepsilon_0 \right)^2} \dfrac{1}{n^{2}}
\end{equation*}

\begin{equation*}
  \psi \left( r, \theta, \varphi \right) = R_{nl} \left( r \right) Y_{lm} \left( \theta, \varphi \right)
\end{equation*}

\subsection{原子电负性}

\begin{equation*}
  \chi = \dfrac{1}{6.3} \left( W_i + W_a \right)
\end{equation*}

电负性低容易失去电子

\subsection{原子间相互作用}

\subsubsection{原子间相互作用势能}

\begin{equation*}
  u \left( r \right) = - \dfrac{A}{r^{m}} + \dfrac{B}{r^{n}}
\end{equation*}

固体中所有原子的势能为

\begin{equation*}
  U = \dfrac{1}{2} \sum_{i,j}^N u \left( r_{ij} \right) = \dfrac{N}{2} \sum_j u \left( r_{ij} \right)
\end{equation*}

平衡时电子距离为

\begin{equation*}
  \left. \dfrac{\partial U \left( r \right) }{\partial r} \right|_{r=r_0} = 0
  \Rightarrow
  r_0 = \left( \dfrac{Bn}{Am} \right)^{m-n}
\end{equation*}

结合能为

\begin{equation*}
  W = E_N - U \left( r_0 \right)
\end{equation*}

通常令$E_N=0$

固体的体积为

\begin{equation*}
  V = N \beta r^3
\end{equation*}

平衡体积为

\begin{equation*}
  V_0 = N \beta r_0^3
\end{equation*}

弹性模量

\begin{equation*}
  K = \left. \dfrac{1}{9N\beta r_{0}} \left( \dfrac{\partial^{2} U}{\partial r^2} \right)\right|_{r=r_0}
\end{equation*}

\subsubsection{离子键结合}

一维离子链,正负离子交替排列,第$j$个离子和第$1$个离子间距离为$r_{1j}=a_jr$

\begin{equation*}
  M = \sum_j \pm \dfrac{1}{a_j}
\end{equation*}

\begin{equation*}
  B = \sum_j \dfrac{b}{a_j^n}
\end{equation*}

\begin{equation*}
  U = - N \left[ \dfrac{Me^{2}}{4\pi \varepsilon_{0} r} - \dfrac{B}{r^{n}} \right]
\end{equation*}

$B$,$n$,$M$为常数,马德隆常数

\begin{equation*}
  M = 2 \ln 2
\end{equation*}

\subsubsection{范德瓦尔斯键}

相互作用能

\begin{equation*}
  u \left( r \right) = - \dfrac{A}{r^{6}} + \dfrac{B}{r^{12}} = 4\varepsilon \left[ - \left( \dfrac{\sigma}{r} \right)^{6} + \left( \dfrac{\sigma}{r} \right)^{12} \right] 
\end{equation*}

整个固体的能量

\begin{equation*}
  U \left( r \right) = 2 N \varepsilon \left[ A_{12} \left( \dfrac{\sigma}{r} \right)^{12} - A_6 \left( \dfrac{\sigma}{r} \right)^{6} \right] 
\end{equation*}

\section{晶体结构表述}

\subsection{正格子空间}

\subsubsection{原胞体积}

\begin{equation*}
  \Omega = \vec{a}_1 \cdot \left( \vec{a}_2 \times \vec{a}_3 \right)
\end{equation*}

\subsubsection{格矢}

\begin{equation*}
  \vec{R}_l = l_1 \vec{a}_1 + l_2 \vec{a}_2 + l_3 \vec{a}_3
\end{equation*}

\subsubsection{晶列指数}

\begin{equation*}
  \left[ m,n,p \right] \Rightarrow \vec{R} = m \vec{a}_1 + n \vec{a}_2 + p \vec{a}_3 
\end{equation*}

\subsubsection{晶面指数}
晶面与基矢截距为$a_1/h_1$,$a_2/h_2$,$a_3/h_3$的晶面为
\begin{equation*}
  \left( h_1h_2h_3 \right)
\end{equation*}

\subsection{倒格子空间}

\subsubsection{倒格子定义}

\begin{equation*}
  \begin{aligned}
    \vec{b}_1 = \dfrac{2\pi}{\Omega} \left( \vec{a}_2 \times \vec{a}_3 \right) \\
    \vec{b}_2 = \dfrac{2\pi}{\Omega} \left( \vec{a}_3 \times \vec{a}_1 \right) \\
    \vec{b}_3 = \dfrac{2\pi}{\Omega} \left( \vec{a}_1 \times \vec{a}_2 \right)
  \end{aligned}
\end{equation*}

\subsubsection{倒格矢}

\begin{equation*}
  \vec{K}_h = h_1 \vec{b}_1 + h_2 \vec{b}_2 + h_3 \vec{b}_3
\end{equation*}

\section{晶体的衍射}

\subsection{劳厄衍射方程}

\begin{equation*}
  \vec{R}_l \cdot \left( \vec{k} - \vec{k}_0 \right) = 2\pi n
\end{equation*}

\subsection{布拉格衍射方程}

\begin{equation*}
  2d \sin \theta = n\lambda
\end{equation*}

\subsection{倒格子空间两种衍射方程的表述}

\begin{equation*}
  \vec{k} - \vec{k}_0 = n \vec{K}_h
\end{equation*}

\subsection{布里渊表述}

$\vec{G}$为倒格矢

\begin{equation*}
  \vec{G} \cdot \left( \vec{k} + \dfrac{\vec{G}}{2} \right) = 0
\end{equation*}

所有的倒格矢$\vec{G}$的垂直平分线构成布里渊区

\section{原子振动}

\subsection{单原子链}

\subsubsection{运动方程}

\begin{equation*}
  m \dfrac{\md^2 x_n}{\md t^2} = \beta \left( x_{n+1} + x_{n-1} - 2x_n \right)
\end{equation*}

令$\omega=\left( 2\beta/m \right)^{1/2}$,方程的解为

\begin{equation*}
  x=A\exp \left[ - i \left( \omega t - qna \right) \right]
\end{equation*}

\subsubsection{格波}

\begin{equation*}
  \lambda = \dfrac{2\pi}{q} \quad\quad\quad v_p = \dfrac{\omega}{q}
\end{equation*}

\subsubsection{色散关系}

把运动方程的解代入运动方程,得到

\begin{equation*}
  \omega = 2 \sqrt{\dfrac{\beta}{m}} \left| \sin \dfrac{qa}{2} \right|  
\end{equation*}

\subsubsection{长波极限}

当$\lambda \rightarrow \infty$时

\begin{equation*}
  \omega = qa \sqrt{\dfrac{\beta}{m}} 
\end{equation*}

弹性模量$K$,介质密度$a$

\begin{equation*}
  a \sqrt{\dfrac{\beta}{m}} = \sqrt{\dfrac{K}{\rho}} = \sqrt{\dfrac{\beta a}{m/a}}  
\end{equation*}

\subsubsection{短波极限}

当$qa/2 \rightarrow \pm \pi /2$时

\begin{equation*}
  v_g = \dfrac{\md \omega}{\md q} = 0 \quad\quad\quad v_p = \dfrac{\omega}{q} = \dfrac{2}{q} \sqrt{\dfrac{\beta}{m}} 
\end{equation*}

\subsubsection{波恩-卡门周期性条件}

$N$个原子组成的原子链

\begin{equation*}
  qNa = \pi l
\end{equation*}

$l$是整数,一个布里渊区里面

\begin{equation*}
  - \dfrac{\pi}{a} < q \leq \dfrac{\pi}{a} \Rightarrow - \dfrac{N}{2} < l \leq \dfrac{N}{2}
\end{equation*}

\subsection{双原子链}

\subsubsection{运动方程}

\begin{equation*}
  \begin{aligned}
    m \dfrac{\md^2 x_{2n+1}}{\md t^2} &= \beta \left( x_{2n+2} + x_{2n} - 2 x_{2n+1} \right) \\
    M \dfrac{\md^2 x_{2n+2}}{\md t^2} &= \beta \left( x_{2n+3} + x_{2n+1} - 2 x_{2n+2} \right)
  \end{aligned}
\end{equation*}

解为

\begin{equation*}
  \begin{aligned}
    x_{2n+1} &= A \exp \left[ i \left( q \left( 2n+1 \right)a-\omega t \right) \right] \\
    x_{2n+2} &= B \exp \left[ i \left( q \left( 2n+2 \right)a-\omega t \right) \right]
  \end{aligned}
\end{equation*}

\subsubsection{色散关系}

将解代入方程,得到

\begin{equation*}
  \left|
    \begin{aligned}
      2\beta - m \omega^2 && -2\beta \cos \left( qa \right) \\
      -2\beta \cos \left( qa \right) && 2\beta -M\omega^2
    \end{aligned}
  \right| =0 
\end{equation*}

解出$\omega_+$为光学波,$\omega_-$为声学波

\subsubsection{波矢取值范围}

\begin{equation*}
  2qNa=2\pi l
\end{equation*}

$l$是整数,将$q$限制在

\begin{equation*}
  - \dfrac{\pi}{2a} < q \leq \dfrac{\pi}{2a}
\end{equation*}

$l$范围

\begin{equation*}
  - \dfrac{N}{2} < l \leq \dfrac{N}{2}
\end{equation*}

\subsection{原子振动的量子理论}

\subsubsection{声子}

能量

\begin{equation*}
  \hbar \omega
\end{equation*}

动量

\begin{equation*}
  \hbar \vec{q}
\end{equation*}

频率为$\omega$的声子数目

\begin{equation*}
  n \left( \omega \right) = \dfrac{1}{\me^{\hbar \omega / k_B T} - 1}
\end{equation*}

\subsubsection{格波能量}

\begin{equation*}
  \varepsilon_{qs} = \left( n_{qs} + \dfrac{1}{2} \right) \hbar \omega_s \left( \vec{q} \right)
\end{equation*}

\subsection{晶格比热的模型}

\subsubsection{爱因斯坦模型}

\begin{equation*}
  C_V = 3Nk_B f_E \left( \dfrac{\theta_E}{T} \right)
\end{equation*}

其中

\begin{equation*}
  \begin{aligned}
    f_E \left( \dfrac{\theta_E}{T} \right) &= \left( \dfrac{\theta_E}{T} \right)^2 \dfrac{\me^{\dfrac{\theta_E}{T}}}{\left( \me^{\dfrac{\theta_E}{T}} -1 \right)^2} \\
    \theta_E &= \dfrac{\hbar \omega_E}{k_B}
  \end{aligned}
\end{equation*}

\subsubsection{德拜模型}

\begin{equation*}
  C_V = 3Nk_B f_D \left( \dfrac{T}{\theta_D} \right)
\end{equation*}

其中

\begin{equation*}
  \begin{aligned}
    f_D \left( T / \omega \right) &= 3 \left( \dfrac{T}{\theta_D}  \right)^3 \int_0^{\theta_D / T} \dfrac{\xi^4 \me^{\xi}}{\left( \me^{\xi} - 1 \right)^2} \md \xi \\
    \xi &= \dfrac{\hbar \omega_m}{k_B T}
  \end{aligned}
\end{equation*}

声子的分布函数为

\begin{equation*}
  g \left( \omega \right) = \dfrac{3 V}{2\pi^2 \bar{v}^3} \omega^2
\end{equation*}

声子频率上限为

\begin{equation*}
  \omega_m = \bar{v} \left[ 6\pi^2 \left( \dfrac{N}{V} \right) \right]^{1/3}
\end{equation*}

\section{金属电子论}

\subsection{电子的能量}

\begin{equation*}
  \varepsilon_n = \dfrac{h^2}{2mL^2} \left( n_x^2 + n_y^2 + n_z^2 \right) = \dfrac{\hbar}{2m} \left( k_x^2 + k_y^2 + k_z^2 \right) 
\end{equation*}

\subsection{状态空间}

\begin{equation*}
  \dfrac{1}{\Delta \vec{k}} = \dfrac{V}{8\pi^3}
\end{equation*}

\subsection{能态密度}

能量为$\varepsilon$的电子在一个半径为

\begin{equation*}
  k = \dfrac{\sqrt{2m\varepsilon} }{\hbar}
\end{equation*}

的球面上,能量处于$\varepsilon$到$\varepsilon + \md \varepsilon$之间电子数目为

\begin{equation*}
  \md N = 2\cdot \dfrac{V}{8\pi^3}\cdot 4\pi k^2 \md k = 4\pi V \left( \dfrac{2m}{h^2} \right)^{3/2} \varepsilon^{1/2} \md \varepsilon
\end{equation*}

能态密度

\begin{equation*}
  g \left( \varepsilon \right) = \dfrac{\md N}{\md \varepsilon} = C \varepsilon^{1/2}
\end{equation*}

其中

\begin{equation*}
  C = 4\pi V \left( \dfrac{2m}{h^2} \right)^{3/2}
\end{equation*}

\subsection{费米面}

\begin{equation*}
  nV = 2\cdot \dfrac{V}{8\pi^3}\cdot \dfrac{4}{3} \pi \cdot \left( k_F^0 \right)^3 \Rightarrow k_F^0 = \left( 3\pi^2 n \right)^{1/3}
\end{equation*}

\subsection{费米面相关的物理量}

费米能

\begin{equation*}
  \varepsilon_F^0 = \dfrac{\hbar^2 \left( k_F^0 \right)^2}{2m}
\end{equation*}

费米动量

\begin{equation*}
  p_F^0 = \hbar k_F^0
\end{equation*}

费米速度

\begin{equation*}
  v_F^0 = \dfrac{\hbar k_F^0}{m}
\end{equation*}

费米温度

\begin{equation*}
  T_F^0 = \dfrac{\varepsilon_F^0}{k_B}
\end{equation*}

基态能量

\begin{equation*}
  E_0 = 2\cdot \dfrac{V}{8\pi^3} \int_0^{k_F^0} \dfrac{\hbar^2 k^2}{2m} \cdot 4\pi k^2 \md k
\end{equation*}

\subsection{费米-狄拉克统计}

\begin{equation*}
  f \left( \varepsilon \right) = \dfrac{1}{\me^{\left( \varepsilon - \varepsilon_F \right)/k_B T} + 1}
\end{equation*}

积分近似公式

\begin{equation*}
  \int_0^{\infty} G \left( \varepsilon \right) \left( - \dfrac{\partial f}{\partial \varepsilon} \right) \md \varepsilon = G \left( \varepsilon_F \right) + \dfrac{\pi^2}{6} \left( k_B T \right)^2 G'' \left( \varepsilon_F \right)
\end{equation*}

通常遇到积分的式子,先分布积分,之后使用这个公式

\subsection{费米能随温度的变化关系}

\begin{equation*}
  \varepsilon_F = \varepsilon_F^0 \left[ 1 - \dfrac{\pi^2}{12} \left( \dfrac{k_B T}{\varepsilon_F^0} \right)^2 \right]
\end{equation*}

\subsection{金属自由电子气的能量}

\begin{equation*}
  E = \dfrac{3}{5} N \varepsilon_F^0 \left[ 1 + \dfrac{5\pi^2}{12} \left( k_B T / \varepsilon_F^0 \right)^2 \right]
\end{equation*}

\subsection{金属自由电子气的比热}

\begin{equation*}
  C_V^e = \dfrac{\partial E}{\partial T} = \dfrac{\pi^2}{2} Nk_B \left( k_BT / \varepsilon_F^0 \right)
\end{equation*}

\section{固体能带论}

\subsection{固体s态电子的能带}

\begin{equation*}
  \varepsilon_s \left( \vec{k} \right) = \varepsilon_s^{\alpha t} - J \left( 0 \right) - J \sum \exp \left[ i \left( \vec{k} \cdot \vec{R}_n \right) \right]
\end{equation*}

求和是对所有进邻原子,也就是挨着的原子,求和

\subsection{电子有效质量}

\begin{equation*}
  m^{*} = \dfrac{\hbar^2}{\dfrac{\partial^2 \varepsilon_s}{\partial k^2}}
\end{equation*}

\subsection{电子平均速度}

\begin{equation*}
  \vec{v} \left( \vec{k} \right) = \dfrac{1}{\hbar} \nabla_{\vec{k}} \varepsilon \left( \vec{k} \right)
\end{equation*}

\end{multicols}

\end{document} 

%%% Local Variables:
%%% mode: latex
%%% TeX-master: t
%%% End:
