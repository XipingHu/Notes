\documentclass{article}

% Chinese Support using xeCJK
% \usepackage{xeCJK}
% \setCJKmainfont{SimSun}

% Chinese Support using CTeX
\usepackage{ctex}

% Math Support
\usepackage{amsmath}
\usepackage{amsfonts}
\usepackage{amssymb}
\usepackage{wasysym}
\newcommand{\angstrom}{\text{\normalfont\AA}}

\usepackage{fancyhdr}

% Graphics Support
\usepackage{graphicx}
\usepackage{float}
\restylefloat{table}

% Reduced page margin
\usepackage{geometry}
\geometry{a4paper,scale=0.8}

\usepackage{caption}
\usepackage{subcaption}

% d and e should be math operators
\newcommand*{\dif}{\mathop{}\!\mathrm{d}}
\newcommand*{\md}{\mathop{}\!\mathrm{d}}
\newcommand*{\me}{\mathrm{e}}

% No indent for each paragraph
\usepackage{parskip}
\setlength{\parindent}{0cm}

% Bold style for Greek letters
\usepackage{bm}
\let\Oldmathbf\mathbf
\renewcommand{\mathbf}[1]{\boldsymbol{\Oldmathbf{#1}}}

% More space for dfrac in cell
\usepackage{cellspace}
\setlength{\cellspacetoplimit}{5pt}
\setlength{\cellspacebottomlimit}{5pt}

% SI units
\newcommand{\si}[1]{\  \mathrm{#1}}

% Multi-line author information
\usepackage{authblk}
\author{胡喜平}
\usepackage{multicol}
\title{固体物理}

\pagestyle{fancy}
\fancyhf{}
\lhead{https://hxp.plus/}
\rhead{固体物理}
\rfoot{第 \thepage 页}
\renewcommand{\headrulewidth}{1pt}
\renewcommand{\footrulewidth}{1pt}

\begin{document}


\begin{multicols}{2}
  

\section{原子的凝聚}

\subsection{原子结构}

原子的能量和波函数

\begin{equation*}
  E_n = - \frac{\mu Z e^{4}}{2 \hbar^{2} \left( 4 \pi \varepsilon_0 \right)^2} \frac{1}{n^{2}}
\end{equation*}

\begin{equation*}
  \psi \left( r, \theta, \varphi \right) = R_{nl} \left( r \right) Y_{lm} \left( \theta, \varphi \right)
\end{equation*}

\subsection{原子电负性}

\begin{equation*}
  \chi = \frac{1}{6.3} \left( W_i + W_a \right)
\end{equation*}

电负性低容易失去电子

\subsection{原子间相互作用}

\subsubsection{原子间相互作用势能}

\begin{equation*}
  u \left( r \right) = - \frac{A}{r^{m}} + \frac{B}{r^{n}}
\end{equation*}

固体中所有原子的势能为

\begin{equation*}
  U = \frac{1}{2} \sum_{i,j}^N u \left( r_{ij} \right) = \frac{N}{2} \sum_j u \left( r_{ij} \right)
\end{equation*}

平衡时电子距离为

\begin{equation*}
  \left. \frac{\partial U \left( r \right) }{\partial r} \right|_{r=r_0} = 0
  \Rightarrow
  r_0 = \left( \frac{Bn}{Am} \right)^{m-n}
\end{equation*}

结合能为

\begin{equation*}
  W = E_N - U \left( r_0 \right)
\end{equation*}

通常令$E_N=0$

固体的体积为

\begin{equation*}
  V = N \beta r^3
\end{equation*}

平衡体积为

\begin{equation*}
  V_0 = N \beta r_0^3
\end{equation*}

弹性模量

\begin{equation*}
  K = \left. \frac{1}{9N\beta r_{0}} \left( \frac{\partial^{2} U}{\partial r^2} \right)\right|_{r=r_0}
\end{equation*}

\subsubsection{离子键结合}

一维离子链,正负离子交替排列,第$j$个离子和第$1$个离子间距离为$r_{1j}=a_jr$

\begin{equation*}
  M = \sum_j \pm \frac{1}{a_j}
\end{equation*}

\begin{equation*}
  B = \sum_j \frac{b}{a_j^n}
\end{equation*}

\begin{equation*}
  U = - N \left[ \frac{Me^{2}}{4\pi \varepsilon_{0} r} - \frac{B}{r^{n}} \right]
\end{equation*}

$B$,$n$,$M$为常数,马德隆常数

\begin{equation*}
  M = 2 \ln 2
\end{equation*}

\subsubsection{范德瓦尔斯键}

相互作用能

\begin{equation*}
  u \left( r \right) = - \frac{A}{r^{6}} + \frac{B}{r^{12}} = 4\varepsilon \left[ - \left( \frac{\sigma}{r} \right)^{6} + \left( \frac{\sigma}{r} \right)^{12} \right] 
\end{equation*}

整个固体的能量

\begin{equation*}
  U \left( r \right) = 2 N \varepsilon \left[ A_{12} \left( \frac{\sigma}{r} \right)^{12} - A_6 \left( \frac{\sigma}{r} \right)^{6} \right] 
\end{equation*}

\section{晶体结构表述}

\subsection{正格子空间}

\subsubsection{原胞体积}

\begin{equation*}
  \Omega = \vec{a}_1 \cdot \left( \vec{a}_2 \times \vec{a}_3 \right)
\end{equation*}

\subsubsection{格矢}

\begin{equation*}
  \vec{R}_l = l_1 \vec{a}_1 + l_2 \vec{a}_2 + l_3 \vec{a}_3
\end{equation*}

\subsubsection{晶列指数}

\begin{equation*}
  \left[ m,n,p \right] \Rightarrow \vec{R} = m \vec{a}_1 + n \vec{a}_2 + p \vec{a}_3 
\end{equation*}

\subsubsection{晶面指数}
晶面与基矢截距为$\frac{a_1}{h_1}$,$\frac{a_2}{h_2}$,$\frac{a_3}{h_3}$的晶面为
\begin{equation*}
  \left( h_1h_2h_3 \right)
\end{equation*}

\subsection{倒格子空间}

\subsubsection{倒格子定义}

\begin{equation}
  \vec{b}_1 = \frac{2\pi}{\Omega} \left( \vec{a}_2 \times \vec{a}_3 \right)
\end{equation}

\begin{equation}
  \vec{b}_2 = \frac{2\pi}{\Omega} \left( \vec{a}_3 \cdot \vec{a}_1 \right)
\end{equation}

\begin{equation}
  \vec{b}_3 = \frac{2\pi}{\Omega} \left( \vec{a}_1 \times \vec{a}_2 \right)
\end{equation}

\subsubsection{倒格矢}

\begin{equation}
  \vec{K}_h = h_1 \vec{b}_1 + h_2 \vec{b}_2 + h_3 \vec{b}_3
\end{equation}

\section{晶体的衍射}

\subsection{劳厄衍射方程}

\begin{equation}
  \vec{R}_l \cdot \left( \vec{k} - \vec{k}_0 \right) = 2\pi n
\end{equation}

\subsection{布拉格衍射方程}

\begin{equation}
  2d \sin \theta = n\lambda
\end{equation}

\subsection{倒格子空间两种衍射方程的表述}

\begin{equation}
  \vec{k} - \vec{k}_0 = n \vec{K}_h
\end{equation}

\subsection{布里渊表述}

$\vec{G}$为倒格矢

\begin{equation}
  \vec{G} \cdot \left( \vec{k} + \frac{\vec{G}}{2} \right) = 0
\end{equation}

所有的倒格矢$\vec{G}$的垂直平分线构成布里渊区

\end{multicols}

\end{document} 

%%% Local Variables:
%%% mode: latex
%%% TeX-master: t
%%% End:
