\chapter{Electromagnetic Theory and photons}

\section{Maxwell's Equation}

Faraday's Induction Law
\begin{equation*}
  \oint_{c} \vec{E} \cdot \md \vec{l} = - \dfrac{\md}{\md t} \iint \vec{B} \cdot \md \vec{S}
\end{equation*}

Gauss's Law
\begin{equation*}
  \oiint_{A} \vec{E} \cdot \md \vec{S} = \dfrac{1}{\varepsilon_{0}} \iiint_{v} \rho \md V 
\end{equation*}

\begin{equation*}
  \oiint_{A} \vec{B} \cdot \md \vec{S} = 0
\end{equation*}

Ampere's Circuital Law

\begin{equation*}
  \oint_{C} \vec{B} \cdot \md \vec{l} = \mu_{0} \iint_{A} \left( \vec{J} + \varepsilon_{0} \dfrac{\partial \vec{E}}{\partial t}  \right) \cdot \md \vec{S}
\end{equation*}

We can now take the derivatives of the 4 equations

\begin{equation*}
  \left\{
  \begin{aligned}
    \nabla \times \vec{E} &= - \dfrac{\partial B}{\partial t} \\
    \nabla \times \vec{B} &= \mu_{0} \varepsilon_{0} \dfrac{\partial \vec{E}}{\partial t}  \\
    \nabla \cdot \vec{E} &= \dfrac{\rho}{\varepsilon_{0}} \\
    \nabla \cdot \vec{B} &= 0
  \end{aligned}
  \right.
\end{equation*}

The Following Equation can be derived from Maxwell Equation above

\begin{equation*}
  \begin{aligned}
    \nabla^{2} \vec{E} = \mu_{0} \varepsilon_{0} \dfrac{\partial^{2} \vec{E}}{\partial t^{2}}\\
    \nabla^{2} \vec{B} = \mu_{0} \varepsilon_{0} \dfrac{\partial^{2} \vec{B}}{\partial t^{2}} 
  \end{aligned}
\end{equation*}

Coincidentally
\begin{equation*}
  c = \dfrac{1}{\sqrt{\mu_{0} \varepsilon_{0}}} 
\end{equation*}

Which indicates the speed of electromagnetic wave is the speed of light.

Furthermore, it can be seen that the electric field and magnetic field are transverse. They are perpendicular to each other. We assume the electric field is parallel to the y-axis.

\begin{equation*}
  \begin{aligned}
    E_{y} (x,t) = E_{0} \cos \left[ \omega \left( t - x/c  \right) + \varepsilon \right]
  \end{aligned}
\end{equation*}
According to Faraday's Law
\begin{equation*}
  \begin{aligned}
    \dfrac{\partial E_{y}}{\partial x} = - \dfrac{\partial B_{Z}}{\partial t}  
  \end{aligned}
\end{equation*}
We can calculate $B_{z}$

\begin{equation*}
  \begin{aligned}
    B_z = \dfrac{1}{c} E_{0} \cos \left[ \omega \left( t - x/c  \right) + \varepsilon \right]
  \end{aligned}
\end{equation*}
So that
\begin{equation*}
  \begin{aligned}
    E_y = c B_z
  \end{aligned}
\end{equation*}
When not in vacuum, similarly
\begin{equation*}
  \begin{aligned}
    E_y=vB_z \\
    v=\frac{1}{\varepsilon\mu}
  \end{aligned}
\end{equation*}

\section{Energy}

\begin{equation*}
  \begin{aligned}
    u_{E} &= \dfrac{\varepsilon_{0}}{2} E^{2} \\
    u_{B} &= \dfrac{1}{2\mu_{0}} B^{2} \\
    u_{E} &= u_{B} \\
    u &= u_{E} + u_{B} = \varepsilon_{0} E^{2} = \dfrac{1}{\mu_{0}} B^{2} \\ 
    S &= uc \\
    \vec{S} &= \dfrac{1}{\mu} \vec{E} \times \vec{B} \quad \mathrm{(Poynting Vector)}\\
    I &= \dfrac{1}{2} \varepsilon v E_0^2
  \end{aligned}
\end{equation*}

\section{Radiation Pressure}

\begin{equation*}
  \begin{aligned}
    P(t) = \dfrac{S(t)}{c} = u = u_E + u_B
  \end{aligned}
\end{equation*}

\begin{equation*}
  \begin{aligned}
    \left< P(t) \right>_T = \dfrac{I}{c} 
  \end{aligned}
\end{equation*}

\begin{equation*}
  \begin{aligned}
    p_V = \dfrac{S}{c^{2}} 
  \end{aligned}
\end{equation*}

\section{Light in Bulk Matter}

\subsection{Speed of light and Dielectric Constant}

\begin{equation*}
  \begin{aligned}
    v = \dfrac{1}{\sqrt{\varepsilon \mu}} 
  \end{aligned}
\end{equation*}
\begin{equation*}
  \begin{aligned}
    n = \dfrac{c}{v} = \sqrt{\dfrac{\varepsilon \mu}{\varepsilon_0 \mu_0} } = \sqrt{\dfrac{\varepsilon}{\varepsilon_0} }
  \end{aligned}
\end{equation*}

\subsection{Dispersion}

For gas and solid

\begin{equation*}
  \begin{aligned}
    m_e \dfrac{\md ^2 x}{\md t^2} + \gamma m_e \dfrac{\md x}{\md t} + m_e \omega_0^2 x = - e E \left( t \right)
  \end{aligned}
\end{equation*}
\begin{equation*}
  \begin{aligned}
    E \left( t \right) = E_0 \exp \left( - i \omega t \right)
  \end{aligned}
\end{equation*}
Assume
\begin{equation*}
  \begin{aligned}
    x = x_0 \exp \left( - i \omega t \right)
  \end{aligned}
\end{equation*}
We got a solution
\begin{equation*}
  \begin{aligned}
    x_0 \left( \omega_0^2 - \omega^2 - i \gamma \omega \right) = - \dfrac{e E_0}{m_e} 
  \end{aligned}
\end{equation*}
\begin{equation*}
  \begin{aligned}
    x_0 = - \dfrac{e E_0}{m_e \left( \omega_0^2 - \omega^2 - i \gamma \omega \right)} 
  \end{aligned}
\end{equation*}
\begin{equation*}
  \begin{aligned}
    x \left( t \right) = - \dfrac{e E \left( t \right)}{m_e \left( \omega_0^2 - \omega^2 - i \gamma \omega \right)}
  \end{aligned}
\end{equation*}
\begin{equation*}
  \begin{aligned}
    P \left( t \right) = - N e x \left( t \right) = \dfrac{N e^2 E \left( t \right)}{m_e \left( \omega_0^2 - \omega^2 - i \gamma \omega \right)} 
  \end{aligned}
\end{equation*}
\begin{equation*}
  \begin{aligned}
    \varepsilon_r = \dfrac{\varepsilon}{\varepsilon_0} = n^2 = 1 + \dfrac{P}{\varepsilon_0 E} = 1 + \dfrac{N e^2}{\varepsilon_0 m_e \left( \omega_0^2 - \omega^2 - i \gamma \omega \right)}  
  \end{aligned}
\end{equation*}
\begin{equation*}
  \left\{
  \begin{aligned}
    \mathrm{Re} \left( \varepsilon_{r} \right) &= 1 + \dfrac{N e^2 \left( \omega_0^2 - \omega^2 \right)}{\varepsilon_0 m_e \left[ \left( \omega_0^2 - \omega^2 \right)^2 + \gamma^2 \omega^2 \right]}  \\
    \mathrm{Im} \left( \varepsilon_{r} \right) &= \dfrac{N e^2 \gamma \omega}{\varepsilon_0 m_e \left[ \left( \omega_0^2 - \omega^2 \right)^2 + \gamma^2 \omega^2 \right]}  
  \end{aligned}
  \right.
\end{equation*}
When $\gamma = 0$
\begin{equation*}
  \begin{aligned}
    \varepsilon_r = n^2 = 1 + \dfrac{N e^2}{\varepsilon_0 m \left( \omega_0^2 - \omega^2 \right)} 
  \end{aligned}
\end{equation*}

For metal

\begin{equation*}
  \begin{aligned}
    m_e \dfrac{\md ^2 x}{\md t^2} + \gamma m_e \dfrac{\md x}{\md t} = - e E \left( t \right)
  \end{aligned}
\end{equation*}

\begin{equation*}
  \begin{aligned}
    \varepsilon_r = 1 - \dfrac{N e^2}{\varepsilon_0 m_e \left(\omega^2 + i \gamma \omega \right)} = 1 - \dfrac{\omega_p^2}{\omega \left( \omega + i \gamma \right)}  
  \end{aligned}
\end{equation*}








%%% Local Variables:
%%% mode: latex
%%% TeX-master: "main"
%%% End:
