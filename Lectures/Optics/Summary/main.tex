\documentclass{article}
\usepackage{amsmath}
\usepackage{amsfonts}
\usepackage{amssymb}
\usepackage{wasysym}
\usepackage{graphicx}
\usepackage{float}
\usepackage{geometry}
\geometry{a4paper,scale=0.8}
\usepackage{caption}
\usepackage{subcaption}
% \newcommand{\oiint}{\mathop{{\int\!\!\!\!\!\int}\mkern-21mu \bigcirc} {}}
\newcommand*{\dif}{\mathop{}\!\mathrm{d}}
\newcommand*{\md}{\mathop{}\!\mathrm{d}}
\newcommand*{\me}{\mathrm{e}}

\usepackage{parskip}
\setlength{\parindent}{0cm}

\usepackage{bm}
\let\Oldmathbf\mathbf
\renewcommand{\mathbf}[1]{\boldsymbol{\Oldmathbf{#1}}}
\let\eqnarray\align

\author{Xiping Hu}
\title{Optics}

\begin{document}
\maketitle

\section{Introduction}

Bing Wang wangbing@hust.edu.cn

\section{Electromagnetic Theory and photons}

\subsection{Maxwell's Equation}

Faraday's Induction Law
\begin{equation}
  \oint_{c} \vec{E} \cdot \md \vec{l} = - \dfrac{\md}{\md t} \iint \vec{B} \cdot \md \vec{S}
\end{equation}

Gauss's Law
\begin{equation}
  \oiint_{A} \vec{E} \cdot \md \vec{S} = \dfrac{1}{\epsilon_{0}} \iiint_{v} \rho \md V 
\end{equation}

\begin{equation}
  \oiint_{A} \vec{B} \cdot \md \vec{S} = 0
\end{equation}

Ampere's Circuital Law

\begin{equation}
  \oint_{C} \vec{B} \cdot \md \vec{l} = \mu_{0} \iint_{A} \left( \vec{J} + \epsilon_{0} \dfrac{\partial \vec{E}}{\partial t}  \right) \cdot \md \vec{S}
\end{equation}

We can now take the derivatives of the 4 equations

\begin{equation}
  \left\{
  \begin{aligned}
    \nabla \times \vec{E} &= - \dfrac{\partial B}{\partial t} \\
    \nabla \times \vec{B} &= \mu_{0} \epsilon_{0} \dfrac{\partial \vec{E}}{\partial t}  \\
    \nabla \cdot \vec{E} &= \dfrac{\rho}{\epsilon_{0}} \\
    \nabla \cdot \vec{B} &= 0
  \end{aligned}
  \right.
\end{equation}

The Following Equation can be derived from Maxwell Equation above

\begin{equation}
  \begin{aligned}
    \nabla^{2} \vec{E} = \mu_{0} \epsilon_{0} \dfrac{\partial^{2} \vec{E}}{\partial t^{2}}\\
    \nabla^{2} \vec{B} = \mu_{0} \epsilon_{0} \dfrac{\partial^{2} \vec{B}}{\partial t^{2}} 
  \end{aligned}
\end{equation}

Coincidentally
\begin{equation}
  c = \dfrac{1}{\sqrt{\mu_{0} \epsilon_{0}}} 
\end{equation}

Which indicates the speed of electromagnetic wave is the speed of light.

Furthermore, it can be seen that the electric field and magnetic field are transverse. They are perpendicular to each other. We assume the electric field is parallel to the y-axis.

\begin{equation}
  \begin{aligned}
    E_{y} (x,t) = E_{0} \cos \left[ \omega \left( t - x/c  \right) + \epsilon \right]
  \end{aligned}
\end{equation}
According to Faraday's Law
\begin{equation}
  \begin{aligned}
    \dfrac{\partial E_{y}}{\partial x} = - \dfrac{\partial B_{Z}}{\partial t}  
  \end{aligned}
\end{equation}
We can calculate $B_{z}$

\begin{equation}
  \begin{aligned}
    B_z = \dfrac{1}{c} E_{0} \cos \left[ \omega \left( t - x/c  \right) + \epsilon \right]
  \end{aligned}
\end{equation}
So that
\begin{equation}
  \begin{aligned}
    E_y = c B_z
  \end{aligned}
\end{equation}
When not in vacuum, similarly
\begin{equation}
  \begin{aligned}
    E_y=vB_z \\
    v=\frac{1}{\epsilon\mu}
  \end{aligned}
\end{equation}

\subsection{Energy}

\begin{equation}
  \begin{aligned}
    u_{E} &= \dfrac{\epsilon_{0}}{2} E^{2} \\
    u_{B} &= \dfrac{1}{2\mu_{0}} B^{2} \\
    u_{E} &= u_{B} \\
    u &= u_{E} + u_{B} = \epsilon_{0} E^{2} = \dfrac{\mu_{0}}{2} B^{2} \\ 
    S &= uc \\
    \vec{S} &= \dfrac{1}{\mu} \vec{E} \times \vec{B} \\
    I &= \dfrac{1}{2} \epsilon v E_0^2
  \end{aligned}
\end{equation}

\subsection{Radiation Pressure}

\begin{equation}
  \begin{aligned}
    P(t) = \dfrac{S(t)}{c} = u = u_E + u_B
  \end{aligned}
\end{equation}

\begin{equation}
  \begin{aligned}
    \left< P(t) \right>_T = \dfrac{I}{c} 
  \end{aligned}
\end{equation}

\begin{equation}
  \begin{aligned}
    p_V = \dfrac{S}{c^{2}} 
  \end{aligned}
\end{equation}





\end{document}