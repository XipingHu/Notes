\chapter{The Energy and Radiation of Atoms}

\section{Rydberg Constant and Wavelength of Radiation}

Wave number of Hydrogen Atoms:

\begin{equation}
  \begin{aligned}
    \tilde{\nu} = \dfrac{1}{\lambda} = Z^2 R \left[ \dfrac{1}{m^2} - \dfrac{1}{n^2}   \right] 
  \end{aligned}
\end{equation}

Where $Z = 1$

\begin{table*}[h]
  \centering
  \begin{tabular}{|Sc|Sc|Sc|}
    \hline
    Lyman Series & $\tilde{\nu} = R \left[ \dfrac{1}{1^2} - \dfrac{1}{n^2}   \right]$ & $n=2,3,4,\dots$ \\
    \hline
    Balmer Series & $\tilde{\nu} = R \left[ \dfrac{1}{2^2} - \dfrac{1}{n^2}   \right]$ & $n=3,4,5,\dots$ \\
    \hline
    Paschen Series & $\tilde{\nu} = R \left[ \dfrac{1}{3^2} - \dfrac{1}{n^2}   \right]$ & $n=4,5,6,\dots$ \\
    \hline
    Brackett Series & $\tilde{\nu} = R \left[ \dfrac{1}{4^2} - \dfrac{1}{n^2}   \right]$ & $n=5,6,7,\dots$ \\
    \hline
    Pfund Series & $\tilde{\nu} = R \left[ \dfrac{1}{5^2} - \dfrac{1}{n^2}   \right]$ & $n=6,7,8,\dots$ \\
    \hline
  \end{tabular}
\end{table*}

Spectroscopic term

\begin{equation*}
  \begin{aligned}
    T(m) = \dfrac{Z^2 R}{m^2} \quad
    T(n) = \dfrac{Z^2 R}{n^2}  
  \end{aligned}
\end{equation*}

Then

\begin{equation}
  \begin{aligned}
    \tilde{\nu} = T \left( m \right) - T \left( n \right)   
  \end{aligned}
\end{equation}

Energy of emitted light

\begin{equation*}
  \begin{aligned}
    E = \dfrac{h c}{\lambda} = h c \tilde{v} = hc \left[ T (m) - T (n) \right] = \dfrac{R h c}{m^2} - \dfrac{R h c}{n^2}  
  \end{aligned}
\end{equation*}

\section{Bohr's Theory of Hydrogen Atoms}

Energy of steady states:

\begin{equation*}
  \begin{aligned}
    E_n = - \dfrac{1}{2} \dfrac{Z e^2}{4 \pi \varepsilon_0 r_n}  
  \end{aligned}
\end{equation*}

Transition

\begin{equation*}
  \begin{aligned}
    h \nu = E_n - E_m
  \end{aligned}
\end{equation*}

Angular Momentum

\begin{equation*}
  \begin{aligned}
    L = n \cdot \dfrac{h}{2 \pi} = n \hbar = m_e v_n r_n
  \end{aligned}
\end{equation*}
\begin{equation*}
  \begin{aligned}
    m_e \dfrac{v_n^2}{r_n} = \dfrac{Z e^2}{4 \pi \varepsilon_0 r_n^2}  
  \end{aligned}
\end{equation*}

So that

\begin{equation*}
  \begin{aligned}
    \dfrac{n^2 \hbar^2}{m_e r_n^3} = \dfrac{Z e^2}{4 \pi \varepsilon_0 r_n^2}
  \end{aligned}
\end{equation*}

From which we can indicate that the radius is quantized.

\begin{equation}
  \begin{aligned}
    r_n = \dfrac{4 \pi \varepsilon_0 \hbar^2}{m_e e^2} \cdot \dfrac{n^2}{Z} = a_0 \cdot \dfrac{n^2}{Z} \quad \left( a_0 = \dfrac{4 \pi \varepsilon_0 \hbar}{m_e e^2}  \right)
  \end{aligned}
\end{equation}

For hydrogen atoms, $Z = 1$. The energy is also quantized.

\begin{equation*}
  \begin{aligned}
    E_n = - \dfrac{1}{2} \dfrac{Z e^2}{4 \pi \varepsilon_0} \dfrac{m_e e^2}{4 \pi \varepsilon_0 \hbar^2} \cdot \dfrac{Z}{n^2} = - \dfrac{m_e e^4}{2 \left( 4 \pi \varepsilon_0 \hbar \right)^2} \cdot \dfrac{Z^2}{n^2} 
  \end{aligned}
\end{equation*}

The velocity is also qumntumized

\begin{equation*}
  \begin{aligned}
    v_n = \dfrac{n \hbar}{m_e r_n} = \dfrac{n \hbar}{m_e}  \dfrac{m_e e^2}{4 \pi \varepsilon_0 \hbar^2} \cdot \dfrac{Z}{n^2} = \dfrac{e^2}{4 \pi \varepsilon_0 \hbar} \cdot \dfrac{Z}{n} =
\dfrac{Z \alpha c}{n} 
  \end{aligned}
\end{equation*}
\begin{equation*}
  \begin{aligned}
    \alpha = \dfrac{e^2}{4 \pi \varepsilon_0 \hbar c} = \dfrac{1}{137}  
  \end{aligned}
\end{equation*}

Calculate the value of Rydberg Constant

\begin{equation*}
  \begin{aligned}
    E = - \dfrac{m_e e^4}{2 \left( 4 \pi \varepsilon_0 \hbar \right)^2} \cdot \dfrac{Z^2}{n^2} = - \dfrac{2 \pi^2 m_e e^4}{\left( 4 \pi \varepsilon_0 h \right)^2} \cdot \dfrac{Z^2}{n^2}
  \end{aligned}
\end{equation*}
\begin{equation*}
  \begin{aligned}
    \dfrac{h c}{\lambda} = E_2 - E_1 =  \dfrac{2 \pi^2 m_e e^4 Z^2}{\left( 4 \pi \varepsilon_0 h \right)^2} \left[ \dfrac{1}{n_1^2} - \dfrac{1}{n_2^2}   \right]
  \end{aligned}
\end{equation*}
\begin{equation*}
  \begin{aligned}
    \tilde{\nu} = \dfrac{1}{\lambda} = \dfrac{2 \pi^2 m_e e^4 Z^2}{\left( 4 \pi \varepsilon_0 \right)^2 h^3 c} \left[ \dfrac{1}{n_1^2} - \dfrac{1}{n_2^2}   \right]
  \end{aligned}
\end{equation*}

$\Rightarrow$

\begin{equation*}
  \begin{aligned}
    R = \dfrac{2 \pi^2 m_e e^4}{\left( 4 \pi \varepsilon_0 \right)^2 h^3 c}
  \end{aligned}
\end{equation*}

\section{Rydburg Constant of Different Atoms}

In previous sections, we assumed that the nucleus is fixed at a point, with the electron surrounded. But in fact, the nucleus's mass is not infinity, and it moves as well. So that for different atoms, the Rydburg Constants varies. In a two-body system, we define the Reduced Mass of a system as

\begin{equation*}
  \begin{aligned}
    \mu = \dfrac{Mm}{M + m} 
  \end{aligned}
\end{equation*}

And we replace the Reduced Mass with $m_e$

\begin{equation}
  \begin{aligned}
    R_{\infty} &= \dfrac{2 \pi^2 m e^4}{\left( 4 \pi \varepsilon_0 \right)^2 h^3 c} \\
    R_A &= R_{\infty} \cdot \left[ \dfrac{1}{1 + \dfrac{m}{M} } \right]
  \end{aligned}
\end{equation}

\section{Sommerfeld's Quantize Condition}

For any coordinate $q$ and its momentum $p$

\begin{equation*}
  \begin{aligned}
    \oint p \md q = n h
  \end{aligned}
\end{equation*}

holds

\section{Quantized Magneton}

\begin{equation*}
  \begin{aligned}
    \mu = i A \\
    i = \dfrac{e}{\tau} 
  \end{aligned}
\end{equation*}
\begin{equation*}
  \begin{aligned}
    A &= \int_0^{2 \pi} \dfrac{1}{2} r \cdot r \md \phi \\
    &= \dfrac{1}{2} \int_0^{\tau} r^2 \omega \md t \\
    &= \dfrac{1}{2 m} \int_o^{\tau} m r^2 \omega \md t \\
    &= \dfrac{p_{\phi}}{2 m} \tau 
  \end{aligned}
\end{equation*}

Consider that

\begin{equation*}
  \begin{aligned}
    \int_0^{2\pi} p_{\phi} \md \phi = 2\pi p_{\phi} = nh
  \end{aligned}
\end{equation*}

We have

\begin{equation*}
  \begin{aligned}
    \mu = \dfrac{e}{2 m} p_{\phi} = \dfrac{eh}{4 \pi m} \cdot n
  \end{aligned}
\end{equation*}

Let

\begin{equation}
  \begin{aligned}
    \mu_B = \dfrac{eh}{4 \pi m} 
  \end{aligned}
\end{equation}

Then $\mu_B$ is the minimum unit of magneton.

%%% Local Variables:
%%% mode: latex
%%% TeX-master: "main"
%%% End:
