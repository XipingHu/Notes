\documentclass{article}

% Chinese Support using xeCJK
% \usepackage{xeCJK}
% \setCJKmainfont{SimSun}

% Chinese Support using CTeX
\usepackage{ctex}

% Math Support
\usepackage{amsmath}
\usepackage{amsfonts}
\usepackage{amssymb}
\usepackage{wasysym}
\newcommand{\angstrom}{\text{\normalfont\AA}}

% Graphics Support
\usepackage{graphicx}
\usepackage{float}

% Reduced page margin
\usepackage{geometry}
\geometry{a4paper,scale=0.85}

\usepackage{caption}
\usepackage{subcaption}

% d and e should be math operators
\newcommand*{\dif}{\mathop{}\!\mathrm{d}}
\newcommand*{\md}{\mathop{}\!\mathrm{d}}
\newcommand*{\me}{\mathrm{e}}

% No indent for each paragraph
\usepackage{parskip}
\setlength{\parindent}{0cm}

% Bold style for Greek letters
\usepackage{bm}
\let\Oldmathbf\mathbf
\renewcommand{\mathbf}[1]{\boldsymbol{\Oldmathbf{#1}}}

% More space for dfrac in cell
\usepackage{cellspace}
\setlength{\cellspacetoplimit}{2pt}
\setlength{\cellspacebottomlimit}{2pt}

% SI units
\newcommand{\si}[1]{\  \mathrm{#1}}

\usepackage{multicol}

% Multi-line author information
\usepackage{authblk}

\pagestyle{empty}

\begin{document}

\begin{multicols}{2}

\section{粒子的散射}

库仑散射公式 $b = \dfrac{Z e^2}{4 \pi \varepsilon_0} \cdot \dfrac{1}{\dfrac{1}{2} mv^2 } \cdot \cot \dfrac{\theta}{2}   $
\\\\
金属箔 $\dfrac{\md n}{n} = \dfrac{N t A \md \sigma}{A}  $,$\md \sigma = 2 \pi b \md b$
  
\section{量子力学初步}

定态薛定谔方程 $Hu=Eu$,$H = - \dfrac{\hbar^2}{2m} \nabla^2 + V $
\\\\
不确定性原理 $\Delta p \Delta x = \Delta E \Delta t \geq \dfrac{\hbar}{2} $

\section{原子的能级和辐射}

里德伯常数 $R = \dfrac{R_{\infty}}{1 + \dfrac{m}{M} } $
\\\\
\begin{tabular}[H]{@{}|Sl|Sl|}
  \hline
  线系 & 波数(氢原子) \\
  \hline
  赖曼系 & $\tilde{\nu} = R_H \left[ \dfrac{1}{1^2} - \dfrac{1}{n^2}   \right]$ \\
  \hline
  巴耳末系 & $\tilde{\nu} = R_H \left[ \dfrac{1}{2^2} - \dfrac{1}{n^2}   \right]$ \\
  \hline
  帕邢系 & $\tilde{\nu} = R_H \left[ \dfrac{1}{3^2} - \dfrac{1}{n^2}   \right]$ \\
  \hline
  布喇开系 & $\tilde{\nu} = R_H \left[ \dfrac{1}{4^2} - \dfrac{1}{n^2}   \right]$ \\
  \hline
  普丰特系 & $\tilde{\nu} = R_H \left[ \dfrac{1}{5^2} - \dfrac{1}{n^2}   \right]$ \\
  \hline
\end{tabular}

波数 $\tilde{\nu} = Z^2 R \left( \dfrac{1}{m^2} - \dfrac{1}{n^2}   \right)$

\section{原子的精细结构}

\begin{tabular}[H]{@{}|Sl|Sl|Sl|}
  \hline
  线系&别名&跃迁 \\
  \hline
  主线系&主线系&P$\rightarrow$S \\
  \hline
  第二辅线系&锐线系&S$\rightarrow$P \\
  \hline
  第一辅线系&漫线系&D$\rightarrow$P \\
  \hline
  伯格曼系&基线系&F$\rightarrow$D \\
  \hline
\end{tabular}

选择定则

\begin{tabular}[H]{@{}|Sl|Sl|}
  \hline
  LS耦合 & jj耦合 \\
  \hline
  $\Delta S = 0$& $\Delta j_1 = 0, \pm 1$ \\
  \hline
  $\Delta L = 0, \pm 1$ & $\Delta j_2 = 0, \pm 1$ \\
  \hline
  $\Delta J = 0, \pm 1 \ \left( 0 \nrightarrow 0 \right)$ & $\Delta J = 0, \pm 1 \ \left( 0 \nrightarrow 0 \right)$ \\
  \hline
\end{tabular}

泡利不相容原理:nsns的三重态不存在

洪特定则:先看S再看L、J,S、L大能级低,J大能级低的是倒转次序

朗德间隔定则:间隔的比例等于较大的J的比

\section{磁场中的原子}

玻尔磁子 $\mu_B = \dfrac{\hbar}{2m} e $

朗德g因子 $g = 1 + \dfrac{- L \left( L + 1 \right) + S \left( S + 1 \right) + J \left( J + 1 \right)}{2 J \left( J + 1 \right)} $

跃迁选择定则

\begin{itemize}
\item $\Delta M = \phantom{\pm} 0$ 产生 $\pi$ 线,沿着磁场方向看不到
  \item $\Delta M = \pm 1$ 产生 $\sigma$ 线,沿着磁场方向看得到
\end{itemize}

跃迁能量 $\Delta E = \left( M_2 g_2 - M_1 g_1\right) \mu_B B$

波数 $\tilde{\nu} = \dfrac{\Delta E}{hc} = \left( M_2 g_2 - M_1 g_1 \right) \dfrac{eB}{4 \pi m c} $

因此定义 $L = \dfrac{eB}{4 \pi m c} $

\section{原子的壳层结构}

对某个$n$,$l$可以取值$0,\dots,n-1$,一共$n$个

对某个$l$,$m_l$可以取值$-l,\dots,+l$,一共$2l+1$个

对某个$m_l$,$m_s$可以取值$-1/2,+1/2$,一共$2$个

$n$壳层一共能容纳$2n^2$个电子

\section{分子物理}

振动光谱(近红外)

$\tilde{\nu} = \dfrac{E_{v2} - E_{v1}}{hc} = \dfrac{f}{c} $ \quad $f = \dfrac{1}{2\pi} \sqrt{\dfrac{K}{M} } $ 

转动光谱(远红外)

$p = \sqrt{J \left( J + 1 \right)} \hbar$ \quad $E_r = \dfrac{1}{2} I \omega^2 = \dfrac{p^2}{2I}  $ 

定义 $B=\dfrac{h}{8 \pi^2 I c} $ 则 $\tilde{\nu} = 2 B J_2$

振动转动光谱(近红外)

$\tilde{\nu} = \left\{
\begin{aligned}
  \tilde{\nu}_0 + 2 B J_2 \quad \Delta J = + 1 \\
  \tilde{\nu}_0 - 2 B J_1 \quad \Delta J = - 1
\end{aligned}
\right.
\quad\quad
\tilde{\nu}_0 = \dfrac{E_{v2} - E_{v1}}{hc} 
$
\end{multicols}

\end{document}